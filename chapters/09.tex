\chapter{Wnioski}
  Niniejsza praca inżynierska poświęcona była zaprojektowaniu i budowie systemu 
  wizualizacji danych dla webowego interfejsu Maszyny W z wykorzystaniem mikrokontrolera 
  \textit{ESP32}. Głównym celem projektu było stworzenie fizycznego modelu, który w czasie 
  rzeczywistym odwzorowuje stan rejestrów i sygnałów sterujących symulatora Maszyny W za 
  pomocą adresowalnych diod LED oraz umożliwia interakcję z nim poprzez przyciski fizyczne.

  \section{Realizacja celów i wymagań}
    Wszystkie kluczowe cele pracy zostały zrealizowane, a system jest w pełni funkcjonalny:
    \begin{itemize}
      
      \item \textbf{Wizualizacja danych} \textemdash{} system skutecznie wyświetla zarówno
        wartości liczbowe oraz stany sygnałów sterujących na tablicy złożonej z ponad 600 
        adresowalnych diod LED RGB;

      \item \textbf{Interakcja fizyczna} \textemdash{} zaimplementowano obsługę 17 przycisków
        oraz enkodera obrotowego;

      \item \textbf{Komunikacja sieciowa} \textemdash{} \textit{ESP32} pełni rolę punktu dostępowego 
        i serwera \textit{HTTP}, zapewniając dwukierunkową komunikację przez protokół \textit{WebSocket};
      
      \item \textbf{Tryb lokalny} \textemdash{} zrealizowano założenie pozwalające na pracę 
        urządzenia w trybie autonomicznym, bez potrzeby połączenia z zewnętrznym komputerem.
      
    \end{itemize}
  
  \clearpage
  \section{Problemy napotkane w trakcie pracy}
    Podczas realizacji projektu napotkano kilka istotnych wyzwań technicznych, które 
    wymagały modyfikacji pierwotnych założeń projektowych:

    \begin{itemize}
      
      \item \textbf{Problem z spadkami napięcia podświetlenia LED} \textemdash{} 
        Podczas testów prototypu stwierdzono, że tranzystory \textit{MOSFET} odpowiedzialne
        za sterowanie napięciem podświetlenia led zastosowane na płycie głównej nie działały
        zgodnie z oczekiwaniami (np. problemy z pełnym otwarciem kanału przy napięciu sterującym 
        z \textit{ESP32} lub nieprawidłowa charakterystyka przełączania). Aby zapewnić stabilne 
        podświetlenie urządzenia podjęto decyzję o rezygnacji ze sterowania nimi z poziomu 
        mikrokontrolera na rzecz stałego podłączenia diod LED do zasilania. Rozwiązanie to 
        zapewnia stabilne podświetlenie, pozbywając użytkownika możliwości sterowania nim.
      
      \item \textbf{Spadki napięcia na linii diod LED RGB} \textemdash{} 
        Spadki napięcia w linii diod LED. Duża liczba diod wymusiła przejście z 
        zasilania 5~V na 12~V (model \textit{WS2815B}), co wyeliminowało błędy w odwzorowaniu barw na 
        końcach linii LED.
      
      \item \textbf{Ograniczona liczba wyprowadzeń ESP32} \textemdash{} Problem ten 
        rozwiązano poprzez zastosowanie 16-kanałowego multipleksera analogowo-cyfrowego,
        co pozwoliło na oszczędność pinów mikrokontrolera.

    \end{itemize} 
    
  \section{Kierunek dalszego rozwoju}
    Projekt SWIM był tworzony z zamysłem dalszego rozwoju jako narzędzie dydaktyczne. Poniższa
    praca dyplomowa służy za dokumentację, a zarazem instrukcję obsługi urządzenia. Podczas
    pracy zanotowano następujące możliwe dalsze kierunki rozwoju projektu:

    \begin{itemize}

      \item \textbf{Poprawa projektu płyty głównej} \textemdash{} kolejna wersja urządzenia powinna 
        uwzględnić problemy z zasilaniem podświetlenia LED. 

      \item \textbf{Poprawa projektu wyświetlaczy} \textemdash{} wybrany model LED posiada
        dodatkową pomocniczą linię danych, która nie jest wykorzystana na wyprowadzeniach wyświetlaczy.
        Rewizja powinna wykorzystywać tę funkcjonalność.

      \item \textbf{Wsparcie dla rozszerzeń modelu dydaktycznego} \textemdash{} Maszyna W w swojej pełnej wersji
        posiada obsługę przerwań, dodatkowe sygnały sterujące oraz dodatkowe rejestry poszerzające
        jej funkcjonalność \cite{bib:cwiczenia-tutaj}.  Kolejna wersja urządzenia mogłaby wspierać 
        rozszerzenia maszyny.

    \end{itemize}
