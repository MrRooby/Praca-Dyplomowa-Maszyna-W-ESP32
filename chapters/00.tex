\subsubsection*{Tytuł pracy} 
\Title

\subsubsection*{Streszczenie}  
  Przedmiotem pracy jest zaprojektowanie, budowa oraz zaprogramowanie fizycznego systemu 
  wizualizacji danych dla internetowego symulatora Maszyny W. Urządzenie oparto na 
  mikrokontrolerze ESP32, który pełni rolę serwera i komunikuje się ze stroną internetową
  poprzez protokół WebSocket, zapewniając synchronizację w czasie rzeczywistym. 
  Część sprzętowa obejmuje autorskie projekty płytek drukowanych, wykorzystujące adresowalne
  diody LED WS2815B do prezentacji stanów rejestrów, pamięci i magistral. Interfejs sterujący wzbogacono o 
  17 mechanicznych przycisków oraz enkoder, pozwalających na pełną, dwukierunkową interakcję z modelem.
  Dodatkowo urządzenie posiada lokalny tryb działania, w którym autonomicznie symuluje Maszynę W
  bezpośrednio na mikrokontrolerze. Gotowy system stanowi nowoczesne narzędzie dydaktyczne, ułatwiające 
  zrozumienie architektury komputerów poprzez połączenie symulacji cyfrowej z modelem 
  fizycznym.

\subsubsection*{Słowa kluczowe} 
  Maszyna W, ESP32, Systemy wbudowane, Płytki drukowane, Diody LED RGB, Websocket

\clearpage
\subsubsection*{Thesis title} 
\begin{otherlanguage}{british}
\TitleAlt
\end{otherlanguage}

\subsubsection*{Abstract} 
\begin{otherlanguage}{british}
The subject of this thesis is the design, construction, and programming of a physical data 
visualization system for a web-based simulator of the W Machine. The device is based on the 
ESP32 microcontroller, which acts as a server and communicates with the website via the 
WebSocket protocol, ensuring real-time synchronization. The hardware component includes custom 
PCB designs utilizing addressable WS2815B LEDs to present the states of registers, memory and buses.
The control interface is enhanced with 17 mechanical buttons and an encoder, allowing for full, 
bi-directional interaction with the model. Additionally, the device features a local operating mode,
in which it autonomously simulates the W Machine directly on the microcontroller. The completed 
system serves as a modern educational tool, facilitating the understanding of computer architecture 
by combining digital simulation with a physical model.
\end{otherlanguage}
\subsubsection*{Key words}  
\begin{otherlanguage}{british}
  W Machine, ESP32, Embedded systems, Printed circuit boards, LED RGB diodes, Websocket
\end{otherlanguage}

