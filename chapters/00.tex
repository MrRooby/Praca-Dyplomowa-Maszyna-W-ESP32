\subsubsection*{Tytuł pracy} 
\Title

\subsubsection*{Streszczenie}  
  Przedmiotem pracy jest zaprojektowanie, budowa oraz zaprogramowanie fizycznego systemu 
  wizualizacji danych dla webowego symulatora Maszyny W. Urządzenie oparto na 
  mikrokontrolerze ESP32, który pełni rolę serwera i komunikuje się ze stroną internetową
  poprzez protokół WebSocket, zapewniając synchronizację w czasie rzeczywistym. 
  Część sprzętowa obejmuje autorskie projekty płytek PCB, wykorzystujące ponad 600 adresowalnych 
  diod LED WS2815B do prezentacji stanów rejestrów i magistral. Interfejs sterujący wzbogacono o 
  17 mechanicznych przycisków oraz enkoder, co pozwala na pełną, dwukierunkową interakcję z modelem.
  Dodatkowo urządzenie posiada lokalny tryb działania, w którym autonomicznie symuluje Maszynę W
  bezpośrednio na mikrokontrolerze. Gotowy system stanowi nowoczesne narzędzie dydaktyczne, ułatwiające 
  zrozumienie architektury komputerów poprzez połączenie symulacji cyfrowej z modelem 
  fizycznym.

\subsubsection*{Słowa kluczowe} 
  Maszyna W, ESP32, Systemy wbudowane, Płytki drukowane, Diody LED RGB, Websocket

\subsubsection*{Thesis title} 
\begin{otherlanguage}{british}
\TitleAlt
\end{otherlanguage}

\subsubsection*{Abstract} 
\begin{otherlanguage}{british}
(Thesis abstract – to be copied into an appropriate field during an electronic submission – in English.)
\end{otherlanguage}
\subsubsection*{Key words}  
\begin{otherlanguage}{british}
(2-5 keywords, separated by commas)
\end{otherlanguage}

