\subsubsection*{Tytuł pracy} 
\Title

\subsubsection*{Streszczenie}  
  Przedmiotem pracy jest zaprojektowanie, budowa oraz zaprogramowanie fizycznego systemu wizualizacji danych dla 
  internetowego symulatora Maszyny W. Urządzenie oparte na mikrokontrolerze ESP32 komunikuje się ze stroną poprzez 
  protokół \textit{WebSocket}, zapewniając synchronizację w czasie rzeczywistym. Warstwa sprzętowa obejmuje trzy autorskie 
  projekty płytek drukowanych: płytę główną oraz moduły wyświetlaczy siedmiosegmentowych w wersji dwu- i 
  trzycyfrowej, wykorzystujące adresowalne diody LED WS2815B. Interfejs sterujący z 17 przyciskami i 
  enkoderem umożliwia pełną, dwukierunkową interakcję z modelem, który posiada także lokalny tryb autonomicznej 
  symulacji. Gotowy system stanowi nowoczesne narzędzie dydaktyczne, łączące symulację cyfrową z modelem fizycznym.

\subsubsection*{Słowa kluczowe} 
  Maszyna W, ESP32, Systemy wbudowane, Płytki drukowane, Diody LED RGB

\subsubsection*{Thesis title} 
\begin{otherlanguage}{british}
\TitleAlt
\end{otherlanguage}

\subsubsection*{Abstract} 
\begin{otherlanguage}{british}
  The subject of this thesis is the design, construction, and programming of a physical data visualization 
  system for the Maszyna W web-based simulator. Based on the ESP32 microcontroller, the device utilizes the  \textit{WebSocket} protocol for real-time synchronization with the website. The hardware layer consists of three 
  custom-designed PCBs: a main board and seven-segment display modules in three- and two-digit versions, 
  all employing addressable WS2815B LEDs. A control interface with 17 buttons and an encoder 
  enables full bi-directional interaction, while a local mode allows for autonomous simulation on the 
  microcontroller. The completed system serves as a modern educational tool, bridging digital simulation 
  with a physical model.
\end{otherlanguage}
\subsubsection*{Key words}  
\begin{otherlanguage}{british}
  W Machine, ESP32, Embedded systems, Printed circuit boards, LED RGB diodes
\end{otherlanguage}

