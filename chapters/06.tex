\chapter{Opis oprogramowania}
\label{ch:06}

Program mikrokontrolera ESP32 został napisany w języku \textit{C++} za pomocą aplikacji
\textit{Visual Studio Code} wykorzystującej, dedykowane programowaniu mikrokontrolerów,
rozszerzenie \textit{PlatformIO}. Kod wynikowy zajmuje około 900kB. 
Mikrokontroler ma do zrealizowania następujące zadania:
\begin{itemize}
  \item obsługa adresowalnych diod LED,
  \item obsługa przycisków, enkodera oraz przełącznika trybów,
  \item obsługa pamięci nieulotnej/systemu plików,
  \item obsługa trybu sieciowego SWIM,
  \item obsługa trybu lokalnego SWIM,
  \item kontrola podświetlenia LED.
\end{itemize}
Kompletny kod źródłowy mikrokontrolera wraz z instrukcją instalacji został zamieszczony w dodatku (TODO Dodatek z kodem źródłowym).

\clearpage
\section{Obsługa adresowalnych diod LED} 
\label{sec:obsluga-led}
  Stworzenie wydajnego oraz łatwego w rozbudowie oprogramowania wymagało zaprojektowania odpowiedniej architektury kodu.
  W tym celu stworzono obiekt \textit{DisplayManager}. Jest on odpowiedzialny za zarządzanie wszystkimi adresowalnymi diodami LED urządzenia.
  Implementacja obejmuje następującą funkcjonalność:
  \begin{itemize}
    \item tworzenie/usuwanie wyświetlaczy,
    \item zmianę koloru poszczególnych elementów,
    \item zadawanie wartości do wyświetlenia elementom,
    \item odświeżanie koloru wszystkich elementów naraz,
    \item wyświetlanie animacji testowych/ładowanie.
  \end{itemize}
  
  
  Podczas tworzenia implementacji dla obsługi diod zdecydowano się na abstrakcyjne rozwiązanie problemu wielu różnych typów wyświetlanych wartości.
  Powstała w tym celu klasa wirtualna \textit{LedElement} pozwala na proste odnalezienie elementu w linii diod LED, dynamiczne
  zarządzanie pamięcią urządzenia oraz tworzenie klas pochodnych dla specyficznych wartości takich jak: sygnał, magistrala, czy pamięć operacyjna. 
  Sposób wyświetlania wartości liczbowych przy użyciu diod LED opisano w rodziale \ref{sec:segment-display}. 
  \\


  Wykorzystanie abstrakcyjnego podejścia przy kreacji architektury oprogramowania pozwoliło w łatwy sposób odizolować i wykryć błędy. Pozwala na łatwą 
  rozbudowę lub modyfikację układu wyświetlaczy. Implementacja menadżera w ostatecznej wersji oprogramowania wywołuje metodę odświeżającą wszystkie 
  diody LED zgodnie z poniższym schematem blokowym (Rys. \ref{fig:display-manager-flow-chart}). Widoczny na rysunku schemat ma na celu prezentację 
  podstawowych funkcji zarządzania diodami LED.

  \clearpage
  \begin{figure}[H]
    \centering
    \includegraphics[width=0.6\textwidth]{./graf/display-manager-flow-chart.pdf}
    \caption{Diagram sekwencji działania menadżera diod LED}
    \label{fig:display-manager-flow-chart}
  \end{figure}

\section{Obsługa wyświetlaczy segmentowych}
\label{sec:segment-display}
  Stworzenie czytelnego oraz łatwego w obsłudze oprogramowania wymagało specjalnego obiektu 
  zarządzającego diodami LED wyświetlającymi wartości liczbowe. Model Maszyny W został 
  skonstruowany z 9 wyświetlaczy 3-cyfrowych oraz 8 wyświetlaczy 2-cyfrowych. Zarządzanie 43 cyframi 
  wymagało stworzenia obiektu przechowującego indeksy diod odpowiedzialnych za daną wartość liczbową. 

  Powyżej opisany problem należało rozwiązać na etapie konstrukcji schematów elektrycznych 
  wyświetlaczy (rozdziały \ref{sec:trzyseg-wysw} oraz \ref{sec:trzyseg-wysw}). Rysunek \ref{fig:segment-display}
  przedstawia kolejność połączenia diod w wyświetlaczu segmentowym. Wyjście ostatniej diody segmentu jest podłączone do
  wejścia pierwszej diody kolejnego segmentu lub elementu wyświetlacza. 
  \begin{figure}[H]
    \centering
    \includegraphics[width=0.65\textwidth]{./graf/segment-display.pdf}
    \caption{Wizualizacja schematu działania wyświetlaczy segmentowych}
    \label{fig:segment-display}
  \end{figure} 

  Obiekt \textit{Segment} obecny w oprogramowaniu SWIM jest odpowiedzialny za zarządzanie diodami LED wchodzącymi
  w skład segmentu. Przechowuje indeksy diod, oraz zawiera implementację metody wyświetlania cyfr. 
  

\section{Obsługa przycisków, enkodera, diod płyty głównej oraz przełącznika trybów}
  Podobnie jak w przypadku obsługi diod LED (rozdział \ref{sec:obsluga-led}), przygotowano rozwiązanie umożliwiające 
  zarządzanie wszystkimi elementami sterującymi. W tym celu utworzono obiekt \textit{HumanInterface} nakładający warstwę abstrakcji na 
  niskopoziomowe operacje związane z obsługą wejść urządzenia. 
  \\

  Obiekt \textit{HumanInterface} jest odpowiedzialny za ustawienie wyprowadzeń mikroprocesora. Przy inizjalizacji ustawia wyprowadzenia jako wyjście
  dla multipleksera, przycisku \textit{takt}, przełącznika trybów oraz enkodera. Inicjalizowane również są wyprowadzenia dla dwóch 
  diod LED obecnych na płycie głównej. 

  Obiekt posiada dedykowane metody do obsługi następujących funkcji urządzenia:
  \begin{itemize}
    \item obrót enkodera,
    \item zwracanie stanu przycisku enkodera,
    \item multipleksowanie przycisków,
    \item zwracanie stanu przycisków,
    \item zwracanie stanu przełącznika trybów,
    \item eliminacja drgań styków,
    \item kontrola diod LED płyty głównej,
    \item kontrola podświetlenia LED modelu.
  \end{itemize}
 

\section{Obsługa pamięci nieulotnej/systemu plików}
  Rozmiar plików strony internetowej serwowanej przez urządzenie oraz potrzeba aktualizacji strony wymagała stworzenia 
  sposobu na zapis plików w pamięci nieulotnej urządzenia.
  Dodatkowo wykorzystanie diod LED RGB pozwala na personalizację kolorów poszczególnych elementów SWIM. Strona internetowa
  posiada opcję ustawienia koloru dla: wyświetlaczy segmentowych, sygnałów oraz magistrali. Aby zapewnić możliwość przechowywania
  wybranej konfiguracji również należało zapisać wartości w pamięci nieulotnej urządzenia.

  Wyżej wymienione powody wymagały wykorzystania zewnętrznej biblioteki \textit{LittleFS}. Pozwala ona stworzyć osobną partycję w 
  pamięci flash urządzenia dla systemu plików. Zapisane w ten sposób dane nie ulegają usunięciu po odłączeniu zasilania urządzenia.


\section{Obsługa trybu sieciowego SWIM}



\section{Obsługa trybu lokalnego SWIM}