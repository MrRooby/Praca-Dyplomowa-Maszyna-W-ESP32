\chapter{Opis oprogramowania}
\label{ch:06}
Program mikrokontrolera ESP32 został napisany w języku \textit{C++} za pomocą aplikacji
\textit{Visual Studio Code} wykorzystującej, dedykowane programowaniu mikrokontrolerów,
rozszerzenie \textit{PlatformIO}. Kod wynikowy zajmuje około 900kB. 
Mikrokontroler ma do zrealizowania następujące zadania:
\begin{itemize}
  \item obsługa adresowalnych diod LED,
  \item obsługa przycisków, enkodera oraz przełącznika trybów,
  \item obsługa pamięci nieulotnej/systemu plików,
  \item obsługa trybu sieciowego SWIM,
  \item obsługa trybu lokalnego SWIM,
  \item kontrola podświetlenia LED.
\end{itemize}
Kompletny kod źródłowy mikrokontrolera wraz z instrukcją instalacji został zamieszczony w dodatku (TODO Dodatek z kodem źródłowym).
\\


Poniższe rozdziały opisują w szczegółowy sposób działanie oprogramowania urządzenia. Rozdziały \ref{sec:obsluga-led},
\ref{sec:hum-inter} oraz \ref{sec:filesystem} opisują działanie elementów oprogramowania funkcjonujących zarówno 
w trybie sieciowym, jak i lokalnym urządzenia. Opis działania owych elementów jest kluczowy do zrozumienia funkcjonowania
obu trybów. Metody zarządzania ww. elementami oraz sposób działania obu trybów SWIM zostały omówione w rozdziałach 
\ref{sec:w-server} oraz \ref{sec:w-local}.

\clearpage
Poniższy schemat (Rys. \ref{fig:schemat-ogolny-kodu}) przedstawia punkt wejściowy programu urządzenia. Rozszerzony opis działania widocznych 
na nim elementów znajduje się w ww. rozdziałach.
\begin{figure}[H]
  \centering
  \includegraphics[width=0.6\textwidth]{./graf/schemat-ogolny-kodu.pdf}
  \caption{Ogólny diagram sekwencji działania oprogramowania}
  \label{fig:schemat-ogolny-kodu}
\end{figure}

\clearpage
\section{Obsługa adresowalnych diod LED} 
\label{sec:obsluga-led}
  Stworzenie wydajnego oraz łatwego w rozbudowie oprogramowania wymagało zaprojektowania odpowiedniej architektury kodu.
  W tym celu stworzono obiekt \textit{DisplayManager}. Jest on odpowiedzialny za zarządzanie wszystkimi adresowalnymi diodami LED urządzenia.
  Implementacja obejmuje następującą funkcjonalność:
  \begin{itemize}
    \item tworzenie/usuwanie wyświetlaczy,
    \item zmianę koloru poszczególnych elementów,
    \item zadawanie wartości do wyświetlenia elementom,
    \item odświeżanie koloru wszystkich elementów naraz,
    \item wyświetlanie animacji testowych/ładowanie.
  \end{itemize}
  
  
  Podczas tworzenia implementacji dla obsługi diod zdecydowano się na abstrakcyjne rozwiązanie problemu wielu różnych typów wyświetlanych wartości.
  Powstała w tym celu klasa wirtualna \textit{LedElement} pozwala na proste odnalezienie elementu w linii diod LED, dynamiczne
  zarządzanie pamięcią urządzenia oraz tworzenie klas pochodnych dla specyficznych wartości takich jak: sygnał, magistrala, czy pamięć operacyjna. 
  Sposób wyświetlania wartości liczbowych przy użyciu diod LED opisano w rodziale \ref{sec:segment-display}. 
  \\


  Wykorzystanie abstrakcyjnego podejścia w architekturze oprogramowania pozwoliło w łatwy sposób odizolować i wykryć błędy. Pozwala na łatwą 
  rozbudowę lub modyfikację układu wyświetlaczy. 
  
  
  \clearpage
  Implementacja menadżera w ostatecznej wersji oprogramowania wywołuje metodę odświeżającą wszystkie 
  diody LED zgodnie z poniższym schematem blokowym (Rys. \ref{fig:display-manager-flow-chart}). Widoczny na rysunku schemat ma na celu prezentację 
  podstawowych funkcji zarządzania diodami LED.

  \begin{figure}[H]
    \centering
    \includegraphics[scale=1]{./graf/display-manager-flow-chart.pdf}
    \caption{Diagram sekwencji działania menadżera diod LED}
    \label{fig:display-manager-flow-chart}
  \end{figure}

\section{Obsługa wyświetlaczy segmentowych}
\label{sec:segment-display}
  Stworzenie czytelnego oraz łatwego w obsłudze oprogramowania wymagało specjalnego obiektu 
  zarządzającego diodami LED wyświetlającymi wartości liczbowe. Model Maszyny W został 
  skonstruowany z 9 wyświetlaczy 3-cyfrowych oraz 8 wyświetlaczy 2-cyfrowych. Zarządzanie 43 cyframi 
  wymagało stworzenia obiektu przechowującego indeksy diod odpowiedzialnych za daną wartość liczbową. 

  Powyżej opisany problem należało rozwiązać na etapie konstrukcji schematów elektrycznych 
  wyświetlaczy (rozdziały \ref{sec:trzyseg-wysw} oraz \ref{sec:trzyseg-wysw}). Rysunek \ref{fig:segment-display}
  przedstawia kolejność połączenia diod w wyświetlaczu segmentowym. Wyjście ostatniej diody segmentu jest podłączone do
  wejścia pierwszej diody kolejnego segmentu lub elementu wyświetlacza. 
  \begin{figure}[H]
    \centering
    \includegraphics[width=0.65\textwidth]{./graf/segment-display.pdf}
    \caption{Wizualizacja schematu działania wyświetlaczy segmentowych}
    \label{fig:segment-display}
  \end{figure} 

  Obiekt \textit{Segment} obecny w oprogramowaniu SWIM jest odpowiedzialny za zarządzanie diodami LED wchodzącymi
  w skład segmentu. Przechowuje indeksy diod, oraz zawiera implementację metody wyświetlania cyfr. 
  

\section{Obsługa przycisków, enkodera, diod płyty głównej oraz przełącznika trybów}
\label{sec:hum-inter}
  Podobnie jak w przypadku obsługi diod LED (rozdział \ref{sec:obsluga-led}), przygotowano rozwiązanie umożliwiające 
  zarządzanie wszystkimi elementami sterującymi. W tym celu utworzono obiekt \textit{HumanInterface} nakładający warstwę abstrakcji na 
  niskopoziomowe operacje związane z obsługą wejść urządzenia. 
  \\

  Obiekt \textit{HumanInterface} jest odpowiedzialny za ustawienie wyprowadzeń mikrokontrolera. Przy inizjalizacji ustawia wyprowadzenia jako wyjście
  dla multipleksera, przycisku \textit{takt}, przełącznika trybów oraz enkodera. Inicjalizowane również są wyprowadzenia dla dwóch 
  diod LED obecnych na płycie głównej. 

  Obiekt posiada dedykowane metody do obsługi następujących funkcji urządzenia:
  \begin{itemize}
    \item obrót enkodera,
    \item zwracanie stanu przycisku enkodera,
    \item multipleksowanie przycisków,
    \item zwracanie stanu przycisków,
    \item zwracanie stanu przełącznika trybów,
    \item eliminacja drgań styków,
    \item kontrola diod LED płyty głównej,
    \item kontrola podświetlenia LED modelu.
  \end{itemize}
 

\section{Obsługa pamięci nieulotnej/systemu plików}
\label{sec:filesystem}
  Rozmiar plików strony internetowej udostępnianej przez urządzenie oraz potrzeba aktualizacji strony wymagała stworzenia 
  sposobu na zapis plików w pamięci nieulotnej urządzenia.
  Dodatkowo wykorzystanie diod LED RGB pozwala na personalizację kolorów poszczególnych elementów SWIM. Strona internetowa
  posiada opcję ustawienia koloru dla: wyświetlaczy segmentowych, sygnałów oraz magistrali. Aby zapewnić możliwość przechowywania
  wybranej konfiguracji również należało zapisać wartości w pamięci nieulotnej urządzenia.
  \\

  Wyżej wymienione powody wymagały wykorzystania zewnętrznej biblioteki \textit{LittleFS}. Pozwala ona stworzyć osobną partycję w 
  pamięci flash urządzenia dla systemu plików. Zapisane w ten sposób dane nie ulegają usunięciu po odłączeniu zasilania urządzenia.
  

  Urządzenie przechowuje w pamięci plik z konfiguracją kolorów w formacie \textit{json}. Podczas uruchamiania urządzenia konfiguracja
  jest odczytywana, a kolory zostają odpowiednio ustawione. Użytkownik ma możliwość nadpisania pliku z konfiguracją wykorzystując
  odpowiednią opcję w ustawieniach strony internetowej udostępnianej przez urządzenie.

\clearpage
\section{Obsługa trybu sieciowego SWIM}
\label{sec:w-server}
  Głównym trybem działania urządzenia jest wizualizacja interfejsu sieciowego symulatora Maszyny W. Powstały w tym celu obiekt
  \textit{W\_Server} jest odpowiedzialny za zarządzanie następującymi funkcjami:
  \begin{itemize}
    \item tworzenie punktu dostępowego,
    \item tworzenie serwera DNS,
    \item udostępnianie strony internetowej,
    \item obsługa komunikacji pomiędzy stroną internetową, a serwerem. 
  \end{itemize}


  Poniższy schemat blokowy (Rys. \ref{fig:schemat-web-mode}) przedstawia uproszczony sposób działania
  trybu sieciowego.
  \begin{figure}[H]
    \centering
    \includegraphics[width=0.7\textwidth]{./graf/web-mode-graph.pdf}
    \caption{Uproszczony diagram sekwencji działania trybu sieciowego}
    \label{fig:schemat-web-mode}
  \end{figure}

  Sposób działania pomija szczegóły zwiazane z inicjalizacją poszczególnych komponentów takich jak: tworzenie serwera DNS,
  czy udostępnianie strony internetowej.


\section{Obsługa trybu lokalnego SWIM}
\label{sec:w-local}
  Dodatkowym trybem działania urządzenia jest symulacja działania Maszyny W bezpośrednio na urządzeniu. Powstały 
  w tym celu obiekt \textit{W\_Local} działa analogicznie do obiektu \textit{W\_Server}. Podobnie jak menadżer 
  trybu sieciowego wykorzystuje obiekty \textit{DisplayManager}, \textit{HumanInterface} oraz \textit{FileSystem}.
  
  Poniższy schemat blokowy (Rys. \ref{fig:schemat-local-mode}) prezentuje uproszczony sposób działania trybu sieciowego.
  \begin{figure}[H]
    \centering
    \includegraphics[width=0.6\textwidth]{./graf/local-mode-graph.pdf}
    \caption{Uproszczony diagram sekwencji działania trybu lokalnego}
    \label{fig:schemat-local-mode}
  \end{figure}

  Sygnały wybierane poprzez naciśnięcie odpowiednich przycisków zostają dodane do listy sygnałów. Obiekt posiada funkcję 
  wykrywającą, czy Maszyna W pozwala na daną kombinację rozkazów. Przykładem wykluczających się wzajemnie rozkazów są: 
  czyt/pisz, wyak/wys lub wyl/wyad. W przypadku wykrycia wciśnięcia niedozwolonego przycisku urządzenie nie doda rozkazu do listy.
  Po wciśnięciu przycisku \textit{takt} obiekt wykona rozkazy z listy w kolejności dodania. 

  
  W celu zapewnienia poprawnego działania trybu lokalnego wprowadzono zabezpieczenie przed przekroczeniem wartości możliwej
  do wyświetlenia na wyświetlaczach dwu oraz trzycyfrowych. W przypadku przekroczenia zadanego limitu liczba zostanie wyrównana
  do największej/najmniejszej możliwej do wyświetlenia.
