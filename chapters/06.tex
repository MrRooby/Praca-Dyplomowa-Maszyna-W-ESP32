\chapter{Opis oprogramowania}
\label{ch:06}

\section{Program mikrokontrolera}
  Program mikrokontrolera ESP32 został napisany w języku \textit{C++} za pomocą aplikacji
  \textit{Visual Studio Code}. Kod wynikowy zajmuje około 900kB.
  Mikrokontroler ma do zrealizowania następujące zadaniami:
  \begin{itemize}
    \item obsługa linii adresowalnych diod LED,
    \item obsługa przycisków, enkodera oraz przełącznika trybów,
    \item obsługa pamięci nieulotnej/systemu plików,
    \item obsługa trybu sieciowego SWIM,
    \item obsługa trybu lokalnego SWIM,
    \item kontrola podświetlenia LED SWIM,
    \item obsługa enkodera.
  \end{itemize}
  Kompletny kod źródłowy mikrokontrolera wraz z instrukcją instalacji został zamieszczony w publicznym
  repozytorium pod adresem: \textit{TODO adres repo}. 

  \subsection{Obsługa linii adresowalnych diod LED}
    \begin{figure}[H]
    \centering
    \includegraphics[width=0.8\textwidth]{./graf/display-manager-uml.pdf}
    \caption{Diagram zależności klas obsługujących diody LED}
    \label{fig:display-manager-uml}
    \end{figure}

    TODO opis części kodu z przykładowymi funkcjami itp

  \subsection{Obsługa przycisków, enkodera oraz przełącznika trybów}
  
  \subsection{Obsługa pamięci nieulotnej/systemu plików}

  \subsection{Obsługa trybu sieciowego SWIM}

  \subsection{Obsługa trybu lokalnego SWIM}
