\chapter{Obsługa}
Obsługa SWIM jest zależna od wybranego trybu działania urządzenia. W poniższych dwóch rozdziałach
opiszę sposób obsługi urządzenia w obu trybach.
  
Po podłączeniu urządzenia do zasilacza 12~V urządzenie uruchomi się w jednym z trybów. Wybór jest zależny
od ustawienia przełącznika trybów (Rozdział \ref{sec:przel_tryb}). Przy zmianie stanu przełącznika urządzenie
automatycznie zainicjuje oraz zmieni swój tryb działania.

% \clearpage
\section{Tryb sieciowy}
  W przypadku wybrania trybu sieciowego:

  \begin{enumerate}

    \item Urządzenie udostępnia użytkownikowi dedykowaną sieć lokalną o nazwie \textit{Maszyna W(eb)}.
    W czasie oczekiwania na połączenie użytkownika do sieci na wyświetlaczach segmentowych modelu 
    wyświetlana będzie animacja ładowania w postaci obracających się okręgów.

    \item Na wybranym przez użytkownika urządzeniu (komputer, telefon, tablet, itp.) należy połączyć się z
    udostępnianą siecią poprzez wpisanie hasła \textit{czytwysweiil}. Poprawne połączenie do sieci skutkuje
    wyłączeniem animacji ładowania oraz wyświetleniem na wyświetlaczach \textit{PaO} adresu IP strony internetowej.
    
    \item \textit{(OPCJONALNE)} Jeżeli urządzenie wybrane przez użytkownika wspiera funkcjonalność portalu
    uwierzytelniającego urządzenie użytkownika powinno wyświetlić powiadomienie o potrzebie zalogowania się
    do sieci. Wybierając tę funkcję użytkownik zostaje przekierowany na stronę internetową symulatora 
    Maszyny W.

    \item Jeżeli urządzenie użytkownika nie wspiera wymienionej w powyższym punkcie funkcjonalności, należy
    wpisać w polu adresowym przeglądarki wyświetlany na \textit{PaO} adres IP.

    \item Poprawne połączenie ze stroną internetową poskutkuje wyświetleniem na wyświetlaczu wartości obecnych 
    na interfejsie strony. W prawym górnym rogu strony znajduje się przycisk informujący o poprawnym połączeniu 
    z SWIM.

    \item Na tym etapie użytkownik ma możliwość sterowania urządzeniem w sposób dowjaki. Wykorzystując:
      \begin{itemize}
        
        \item \textbf{Interfejs strony internetowej} \textemdash{} Zmieniając wartości rejestrów, bądź rozkazów
        ich wartości będą wyświetlane na urządzeniu w czasie rzeczywistym,
        
        \item \textbf{Przyciski SWIM} \textemdash{} Wciśnięte przyciski rozkazów będą odwzorowane na interfejsie 
        strony internetowej oraz wyświetlaczach modelu.

      \end{itemize}

    \item \textit{(OPCJONALNE)} Użytkownik ma możliwość zmiany koloru wybranych elementów SWIM poprzez stronę internetową w 
    następujący sposób:
      \begin{enumerate}
        \item Wybór menu ustawień.
        \item W sekcji \textit{Kolory LED} ustawień. Zadanie wybranym wyświetlaczom wartości koloru.
        \item Wciśnięcie przycisku \textit{Wyślij wszystkie kolory do ESP32}.
      \end{enumerate}
    Po przesłaniu wartości kolorów wyświetlacze SWIM zmienią kolor na wybrany przez użytkownika. Konfiguracja
    kolorów zostaje zapisana w pamięci urządzenia. W przypadku ponownego uruchomienia SWIM zostanie
    odczytana wcześniej zapisana konfiguracja.

  \end{enumerate}

\clearpage
\section{Tryb lokalny}
  W przypadku wybrania trybu lokalnego:
  \begin{enumerate}
    
    \item Urządzenie przechodzi bezpośrednio do symulacji Maszyny W. Wszystkie linie rozkazów zostają 
    wyłączone, a wartości w rejestrach oraz \textit{PaO} wyzerowane.
    
    \item Użytkownik ma możliwość wykonania następujących operacji:
    \begin{itemize}
      
      \item \textbf{Przewijanie wyświetlanego fragmentu \textit{PaO}} \textemdash{} wykonując obrót enkoderem 
        w lewo/prawo zmieni wyświetlany na 4 wierszach \textit{PaO} fragment pamięci operacyjnej.

      \item \textbf{Skok do aktywnego adresu w \textit{PaO}} \textemdash{} wciskając przycisk enkodera
        wyświetlacz \textit{PaO} pokaże fragment pamięci operacyjnej zawierający adres zapiany w rejestrze
        \textit{A}.

      \item \textbf{Tryb wprowadzania} \textemdash{} aktywacja/dezaktywacja trybu następuje poprzez przytrzymanie 
        przez 1~s przycisku enkodera. Aktywny tryb wprowadzania jest sygnalizowany miganiem jednego z rejestrów.
        Obrót enkodera w lewo/prawo zmniejsza/zwiększa wartość w wybranym (migającym) rejestrze. 
        Wciśnięcie przycisku enkodera spowoduje zmianę wybranego wyświetlacza.

      \item \textbf{Aktywacja rozkazów} \textemdash{} aktywacja/dezaktywacja rozkazów następuje poprzez
        wciśnięcie odpowiedniego przyciska rozkazu. Aktywacja jest sygnalizowana podświetleniem linii rozkazu.
        Wybrany rozkaz zostaje dodany do listy rozkazów do wykonania w takcie. Wybór niedozwolonej kombinacji
        rozkazów skutkuje brakiem aktywacji rozkazu.

      \item \textbf{Wykonanie taktu} \textemdash{} wciśnięcie przycisku \textit{TAKT} skutkuje wykonaniem
        wybranych przez użytkownika rozkazów. Przypisane im operacje (\textit{czyt, wys, wei}, itp.) zostają 
        wykonane w kolejności ich wyboru.

    \end{itemize}

  \end{enumerate}
