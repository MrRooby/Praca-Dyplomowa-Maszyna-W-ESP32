\chapter{Podsumowanie}
\section{Uruchamianie układu}
  Proces uruchamiania urządzenia przebiegał wieloetapowo, obejmując zarówno warstwę 
  sprzętową jak i programową:
  \begin{itemize}

    \item \textbf{Programowanie mikrokontrolera:} Pierwszym krokiem było przygotowanie mikrokontrolera ESP32 
      do pracy. Program został wgrany z poziomu środowiska \textit{PlatformIO} po uprzednim wprowadzeniu 
      układu w tryb \textit{bootloadera} i podłączeniu go do komputera za pomocą przewodu USB-C.
    
    \item \textbf{Konfiguracja systemu plików:} Ważnym elementem konfiguracji programowej 
      było przygotowanie struktury partycji pamięci flash pod system plików \textit{LittleFS}. 
      Było to niezbędne do poprawnego umieszczenia na urządzeniu plików źródłowych strony 
      internetowej, która stanowi kluczowy element działania urządzenia.
    
    \item \textbf{Rozwiązywanie problemów sprzętowych:} Podczas testów części sprzętowej 
      modelu wystąpiły trudności z poprawnym wyświetlaniem barw przez diody LED wyświetlaczy 
      segmentowych. Część wyświetlaczy \textit{PaO} wyświetlała błędne kolory, bądź nie 
      uruchamiała się wcale. Po przeprowadzonej analizie stwierdzono, że przyczyną były błędy
      w przygotowaniu przewodów łączących poszczególne wyświetlacze, co skutkowało 
      przekłamaniami w przesyłaniu sygnałów sterujących kolorami. 
  
      Problem rozwiązano poprzez wymianę okablowania na nowe, wykonane z dużą precyzją przy 
      użyciu dedykowanej zaciskarki do złącz JST. Zastosowanie profesjonalnego narzędzia 
      zapewniło pewność połączeń elektrycznych i wyeliminowało błędy w odwzorowaniu barw.
    
    \item \textbf{Korekta oprogramowania:} Podczas testów oprogramowania modelu wystąpiły 
      kolejne przekłamania kolorów, tym razem na wyświetlaczach sygnałów sterujących oraz 
      magistral. Test zakładał zadanie każdej pojedynczej diodzie LED koloru czerwonego. 
      Wszystkie wyświetlacze segmentowe wyświetlały kolor czerwony, ale pozostałe elementy 
      świeciły w kolorze zielonym. Po przeprowadzeniu analizy oprogramowania stwierdzono, 
      że przyczyną przekłamań są paski LED, które wykorzystują inną wersję diod 
      WS2815B w układzie GRB w przeciwieństwie do tych wykorzystywanych 
      w wyświetlaczach.

    \item \textbf{Weryfikacja komuniakcji:} Ostatnim etapem było sprawdzenie nawiązywania 
      połączenia przez protokół \textit{WebSocket}, co pozwoliło na synchronizację modelu fizycznego z 
      symulatorem internetowym w czasie rzeczywistym.

  \end{itemize}

\section{Wnioski}
  Niniejsza praca inżynierska poświęcona była zaprojektowaniu i budowie systemu 
  wizualizacji danych dla webowego interfejsu Maszyny W z wykorzystaniem mikrokontrolera 
  ESP32. Głównym celem projektu było stworzenie fizycznego modelu, który w czasie 
  rzeczywistym odwzorowuje stan rejestrów i sygnałów sterujących symulatora Maszyny W za 
  pomocą adresowalnych diod LED oraz umożliwia interakcję z nim poprzez przyciski fizyczne.

  \subsection{Realizacja celów i wymagań}
    Wszystkie kluczowe cele pracy zostały zrealizowane, a system jest w pełni funkcjonalny:
    \begin{itemize}
      
      \item \textbf{Wizualizacja danych} \textemdash{} system skutecznie wyświetla zarówno
        wartości liczbowe oraz stany sygnałów sterujących na tablicy złożonej z ponad 600 
        adresowalnych diod LED RGB;

      \item \textbf{Interakcja fizyczna} \textemdash{} zaimplementowano obsługę 17 przycisków
        oraz enkodera obrotowego;

      \item \textbf{Komunikacja sieciowa} \textemdash{} ESP32 pełni rolę punktu dostępowego 
        i serwera HTTP, zapewniając dwukierunkową komunikację przez protokół \textit{WebSocket};
      
      \item \textbf{Tryb lokalny} \textemdash{} zrealizowano założenie pozwalające na pracę 
        urządzenia w trybie autonomicznym, bez potrzeby połączenia z zewnętrznym komputerem.
      
    \end{itemize}
  
  \clearpage
  \subsection{Problemy napotkane w trakcie pracy}
    Podczas realizacji projektu napotkano kilka istotnych wyzwań technicznych, które 
    wymagały modyfikacji pierwotnych założeń projektowych:

    \begin{itemize}
      
      \item \textbf{Problem z spadkami napięcia podświetlenia LED} \textemdash{} 
        Podczas testów prototypu stwierdzono, że tranzystory MOSFET odpowiedzialne
        za sterowanie napięciem podświetlenia led zastosowane na płycie głównej nie działały
        zgodnie z oczekiwaniami (np. problemy z pełnym otwarciem kanału przy napięciu sterującym 
        z ESP32 lub nieprawidłowa charakterystyka przełączania). Aby zapewnić stabilne 
        podświetlenie urządzenia podjęto decyzję o rezygnacji ze sterowania nimi z poziomu 
        mikrokontrolera na rzecz stałego podłączenia diod LED do zasilania. Rozwiązanie to 
        zapewnia stabilne podświetlenie, pozbywając użytkownika możliwości sterowania nim.
      
      \item \textbf{Spadki napięcia na linii diod LED RGB} \textemdash{} 
        Spadki napięcia w linii diod LED. Duża liczba diod wymusiła przejście z 
        zasilania 5~V na 12~V (model WS2815B), co wyeliminowało błędy w odwzorowaniu barw na 
        końcach linii LED.
      
      \item \textbf{Ograniczona liczba wyprowadzeń ESP32} \textemdash{} Problem ten 
        rozwiązano poprzez zastosowanie 16-kanałowego multipleksera analogowo-cyfrowego,
        co pozwoliło na oszczędność wyprowadzeń mikrokontrolera.

    \end{itemize} 
    
  \subsection{Kierunek dalszego rozwoju}
    Projekt SWIM był tworzony z zamysłem dalszego rozwoju jako narzędzie dydaktyczne. Bieżąca
    praca dyplomowa służy za dokumentację, a zarazem instrukcję obsługi urządzenia. Podczas
    pracy zanotowano następujące możliwe dalsze kierunki rozwoju projektu:

    \begin{itemize}

      \item \textbf{Poprawa projektu płyty głównej} \textemdash{} kolejna wersja urządzenia powinna 
        uwzględnić problemy z zasilaniem podświetlenia LED. 

      \item \textbf{Poprawa projektu wyświetlaczy} \textemdash{} wybrany model LED posiada
        dodatkową pomocniczą linię danych, która nie jest wykorzystana na wyprowadzeniach wyświetlaczy.
        Rewizja powinna wykorzystywać tę funkcjonalność.

      \item \textbf{Wsparcie dla rozszerzeń modelu dydaktycznego} \textemdash{} Maszyna W w swojej pełnej wersji
        posiada obsługę przerwań, dodatkowe sygnały sterujące oraz dodatkowe rejestry poszerzające
        jej funkcjonalność \cite{bib:cwiczenia-tutaj}.  Kolejna wersja urządzenia mogłaby wspierać rozszerzenia 
        Maszyny W.

    \end{itemize}

  
