
% SPRAWDZONO ort, styl oraz interp

\chapter{Część konstrukcyjna}
\label{ch:05}

\section{Obwody drukowane}
  Obwody drukowane zostały wykonane na specjalne zamówienie przez zewnętrzną firmę \textit{JLCPCB}.
  Mozaikę ścieżek płytek drukowanych zaprojektowano na bazie schematów z rozdziału \ref{ch:04}. Wykorzystano 
  w tym celu program \textit{KiCAD}.
  Odpowiednie rysunki przedstawiające warstwę górną oraz dolną płytek drukowanych 
  przedstawiono w podrozdziałach \ref{circ:01} oraz \ref{circ:02}.


  \clearpage
  \subsection{Płyta główna}
    \label{circ:01}
    Na rysunkach \ref{fig:plyta-glowna-pcb-front} oraz \ref{fig:plyta-glowna-pcb-rewers} przedstawiono mozaiki obwodów drukowanych
    płyty głównej SWIM.
    \\

    Front płytki zapewnia możliwość przymocowania elementów urządzenia wyłączając wyprowadzenia oraz porty.
    Rysunek \ref{fig:plyta-glowna-pcb-front} przedstawia płytkę pomniejszoną dwukrotnie względem wymiarów rzeczywistych.
    \begin{figure}[H]
      \centering
      \includegraphics[scale=0.5]{./graf/plyta-glowna-pcb-front.pdf}
      \caption{Widok warstwy górnej płyty głównej}
      \label{fig:plyta-glowna-pcb-front}
    \end{figure}

    Rewers płytki zapewnia możliwość przymocowania wyprowadzeń oraz portów urządzenia. Umożliwia również podłączenie
    dodatkowych elementów przy wykorzystaniu dostępnych wyprowadzeń dodatkowych.
    Rysunek \ref{fig:plyta-glowna-pcb-front} przedstawia płytkę pomniejszoną dwukrotnie względem wymiarów rzeczywistych.
    \begin{figure}[H]
      \centering
      \includegraphics[scale=0.5]{./graf/plyta-glowna-pcb-back.pdf}
      \caption{Widok warstwy dolnej płyty głównej}
      \label{fig:plyta-glowna-pcb-rewers}
    \end{figure}



  \clearpage
  \subsection{Wyświetlacze siedmiosegmentowe}
    \label{circ:02}

    Na rysunkach \ref{fig:wysw-trzy-front} oraz \ref{fig:wysw-trzy-rewers} przedstawiono mozaiki obwodów drukowanych
    wyświetlacza trzycyfrowego. 
    \\
    
    Front płytki zapewnia możliwość przymocowania
    diod LED. Rysunek \ref{fig:wysw-trzy-front} przedstawia płytkę w rzeczywistych wymiarach.
    \begin{figure}[H]
    \centering
    \includegraphics[scale=1]{./graf/wysw-trzy-back.pdf}
    \caption{Widok warstwy górnej wyświetlacza trzycyfrowego}
    \label{fig:wysw-trzy-front}
    \end{figure}

    Rewers płytki zapewnia możliwość przymocowania kondensatorów filtrujących.
    Rysunek \ref{fig:wysw-trzy-rewers} przedstawia płytkę w rzeczywistych wymiarach.
    \begin{figure}[H]
    \centering
    \includegraphics[scale=1]{./graf/wysw-trzy-front.pdf}
    \caption{Widok warstwy dolnej wyświetlacza trzycyfrowego}
    \label{fig:wysw-trzy-rewers}
    \end{figure}


    \clearpage
    Na rysunkach \ref{fig:wysw-dwu-front} oraz \ref{fig:wysw-dwu-rewers} przedstawiono mozaiki obwodów drukowanych
    wyświetlacza dwucyfrowego. 
    \\

    Front płytki zapewnia możliwość przymocowania
    diod LED. Rysunek \ref{fig:wysw-dwu-front} przedstawia płytkę w rzeczywistych wymiarach.
    \begin{figure}[H]
    \centering
    \includegraphics[scale=1]{./graf/wysw-dwu-back.pdf}
    \caption{Widok warstwy górnej wyświetlacza dwucyfrowego}
    \label{fig:wysw-dwu-front}
    \end{figure}

    Rewers płytki zapewnia możliwość przymocowania kondensatorów filtrujących.
    Rysunek \ref{fig:wysw-dwu-rewers} przedstawia płytkę w rzeczywistych wymiarach.
    \begin{figure}[H]
    \centering
    \includegraphics[scale=1]{./graf/wysw-dwu-front.pdf}
    \caption{Widok warstwy dolnej wyświetlacza dwucyfrowego}
    \label{fig:wysw-dwu-rewers}
    \end{figure}

\clearpage
\section{Opis złącz}
  \subsection{Płyta główna}

    \subsubsection{Złącze zasilania}
      Urządzenie jest wyposażone w 2-wyprowadzeniową listwę zaciskową typu \textit{ARK}. 
      Złącze pełni funkcję głównego wejścia zasilania i przyjmuje napięcie 12V. 
      Układ elektryczny wyposażono w zabezpieczenie przed odwrotną polaryzacją, 
      uniemożliwiające uszkodzenie płytki w przypadku błędnego podłączenia przewodów.

    \subsubsection{Złącze USB-C}
      W celu umożliwienia łatwego programowania i komunikacji z mikroprocesorem, 
      płytkę wyposażono w złącze USB typu C (tryb USB 2.0 Device).
      Wyprowadzone są jedynie linie \textit{D+} oraz \textit{D−}. Linie zasilania 
      \textit{VBUS} są wykorzystywane do zasilania sekcji USB oraz wykrywania 
      podłączenia komputera. Aby zapewnić zabezpieczenie przed prądem wstecznym
      wykorzystano diodę \textit{Schottky'ego} połączoną w kierynku zaporowym na linii zasilania.
      Pozostałe wyprowadzenia USB-C niewykorzystywane przez układ 
      zostały pozostawione niepodłączone, zgodnie ze specyfikacją.

    \subsubsection{Złącze enkodera}
      Do podłączenia impulsatora zastosowano 5-wyprowadzeniowe złącze typu \textit{JST-ZH} o rastrze 1.5mm.
      Wyprowadzone sygnały:
      \begin{itemize}
        \item \textit{+3V3} -- zasilanie elementu,
        \item \textit{GND} -- masa układu,
        \item \textit{CLK} -- pierwszy kanał kwadraturowy enkodera,
        \item \textit{DT} -- drugi kanał kwadraturowy enkodera, 
        \item \textit{SW} -- złącze przycisku enkodera
      \end{itemize}
      Sygnały prowadzone są do wejść cyfrowych mikroprocesora. Redukcja drgań styków 
      zapewniono z poziomu oprogramowania.

    \subsubsection{Złącza przycisków}
      Każdy przycisk posiada dedykowane 2-wyprowadzeniowe złącze typu \textit{JST-ZH} o rastrze 1.5mm. 
      Jedno wyprowadzenie każdego złącza podłączone jest do masy \textit{GND}, drugie -- do wejścia cyfrowego mikroprocesora.
      Linie sygnałowe wyposażono w rezystor polaryzujący do zasilania z poziomu wbudowanych w mikroprocesor 
      rezystorów. Filtrację sygnałów zapewniono od strony oprogramowania.

    \subsubsection{Złącze przełącznika trybów}
      Przełącznik trybów posiada dedykowane 2-wyprowadzeniowe złącze typu \textit{JST-ZH} o rastrze 1.5mm.
      Jedno wyprowadzenie złącza podłączone jest do masy \textit{GND}, drugie -- do wejścia cyfrowego mikroprocesora 
      wykorzystującego wbudowany w mikroprocesor rezystor polaryzujący do zasilania. 
      Filtrację sygnałów zapewniono od strony oprogramowania.

    \subsubsection{Złącza linii LED}
      Dwa wyprowadzenia linii LED wykorzystują 3-wyprowadzeniowe złącza typu \textit{JST-XH} o rastrze 2.5mm, 
      prowadzące sygnały do linii diod LED typu \textit{WS2815B-V1}.
      Wyprowadzenia:
      \begin{itemize}
        \item \textit{+5V} -- zasilanie diod,
        \item \textit{DATA} -- linia danych,
        \item \textit{GND} -- masa układu.
      \end{itemize}
      Dodatkowo zastosowano kondensator filtrujący do masy na każdej z adresowalnych diod obecnych na płytkach
      drukowanych, zgodnie z dokumentacją producenta.

    \subsubsection{Złącza zasilania podświetlenia modelu}
      Płytka zawiera dodatkowe dwa 2-wyprowadzeniowe złącza typu \textit{JST-XH} o rastrze 2.5mm. 
      Każde złącze wspiera kontrolę zasilania 12V z poziomu mikroprocesora. W tym celu
      zostały wykorzystane tranzystory typu \textit{MOSFET}.
    
    \subsubsection{Złącza dodatkowego wprowadzania zasilania}
      Projekt układu przewiduje potrzebę dodatkowego wprowadzania zasilania na linie adresowalnych diod LED. 
      W tym celu płytkę wyposażono w 9 złącz typu \textit{JST-XH} o rastrze 2.5mm. Każde złącze zapewnia 
      zasilanie 12V.

  \subsection{Wyświetlacze segmentowe}
    \subsubsection{Złącza wejścia/wyjścia}
      Każdy z wyświetlaczy segmentowych (zarówno trzycyfrowe, jak i dwucyfrowe) posiadają dwa złącza \textit{JST\_IN} oraz \textit{JST\_OUT}
      typu \textit{JST-XH} o rastrze 2.5mm. Złącza umożliwiają łączenie wyświetlaczy w ciągłe linie.
      Wyprowadzenia:
      \begin{itemize}
        \item \textit{+5V} -- zasilanie diod,
        \item \textit{DATA\_IN/DATA\_OUT} -- wejście/wyjście linii danych,
        \item \textit{GND} -- masa układu.
      \end{itemize}