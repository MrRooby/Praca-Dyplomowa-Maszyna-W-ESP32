\chapter{Część konstrukcyjna}
\label{ch:05}

\section{Obwody drukowane}
  Obwody drukowane zostały wykonane na specjalne zamówienie przez zewnętrzną firmę \textit{JLCPCB}.
  Mozaikę ścieżek płytek drukowanych zaprojektowano wykorzystując do tego program \textit{KiCAD}.
  Odpowiednie rysunki przedstawiające warstwę górną oraz dolną płytek drukowanych 
  przedstawiono w rozdziale \ref{circ:01} oraz \ref{circ:02}.

  \subsection{Płyta główna}
    \label{circ:01}

    \begin{figure}[H]
    \centering
    \includegraphics[width=0.7\textwidth]{./graf/plyta-glowna-pcb-front.png}
    \caption{Widok warstwy górnej płyty głównej}
    \label{fig:plyta-glowna-pcb-front}
    \end{figure}

    \begin{figure}[H]
    \centering
    \includegraphics[width=0.7\textwidth]{./graf/plyta-glowna-pcb-back.png}
    \caption{Widok warstwy dolnej płyty głównej}
    \label{fig:plyta-glowna-pcb-rewers}
    \end{figure}



  \clearpage
  \subsection{Wyświetlacze siedmiosegmentowe}
    \label{circ:02}

    % TRZYCYFROWY
    \begin{figure}[H]
    \centering
    \includegraphics[width=0.55\textwidth]{./graf/wysw-trzy-front.png}
    \caption{Widok warstwy dolnej wyświetlacza trzycyfrowego}
    \label{fig:wysw-trzy-front}
    \end{figure}

    \begin{figure}[H]
    \centering
    \includegraphics[width=0.55\textwidth]{./graf/wysw-trzy-back.png}
    \caption{Widok warstwy dolnej wyświetlacza trzycyfrowego}
    \label{fig:wysw-trzy-rewers}
    \end{figure}

    % DWUCYFROWY
    \begin{figure}[H]
    \centering
    \includegraphics[width=0.4\textwidth]{./graf/wysw-dwu-front.png}
    \caption{Widok warstwy dolnej wyświetlacza dwucyfrowego}
    \label{fig:wysw-dwu-front}
    \end{figure}

    \begin{figure}[H]
    \centering
    \includegraphics[width=0.4\textwidth]{./graf/wysw-dwu-back.png}
    \caption{Widok warstwy dolnej wyświetlacza dwucyfrowego}
    \label{fig:wysw-dwu-rewers}
    \end{figure}

\section{Opis złącz}
  \subsection{Płyta główna}
    \subsubsection{Złącze zasilania}
      Urządzenie jest wyposażone w 2-pinową listwę zaciskową. Złącze zasilania przyjmuje napięcie 12V. Układ 
      elektryczny został wyposażony w zabezpieczenie przed odwrotną polaryzacją podłączanych
      przewodów. Wybrany model złącza wspiera napięcia do 250V oraz prąd znamionowy do 18A. 
  \subsubsection{Złącze USB-C}
      W celu zapewnienia łatwego programowania obecnego na płycie głównej mikroprocesora układ został wyposażony 
      w złącze USB typu C. 


      TODOOO
  \subsubsection{Złącze enkodera}
  \subsubsection{Złącza przycisków}
  \subsubsection{Złącze przełącznika trybów}
  \subsubsection{Złącza linii LED}
  \subsubsection{Złącza zasilania podświetlenia modelu}
  \subsubsection{Złącza dodatkowego wprowadzania zasilania}

\section{Wyświetlacze segmentowe}
  \subsubsection{Złącza wejścia/wyjścia}