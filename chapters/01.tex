\chapter{Wstęp}
\label{ch:wstep}

\section{Cel pracy}
  Celem niniejszej pracy jest opracowanie systemu wizualizacji danych pochodzących z webowego interfejsu Maszyny W 
  z wykorzystaniem mikrokontrolera ESP32-S3.

  Opracowane w ramach pracy dyplomowej urządzenie umożliwia odwzorowanie w czasie rzeczywistym wartości 
  liczbowych oraz sygnałów sterujących symulatorem Maszyny W na adresowalnych diodach RGB. 
  Ponadto system pozwala na zdalne sterowanie funkcjami symulatora zarówno za pomocą przycisków 
  fizycznych, jak i interfejsu internetowego. System obsługuje hosting interfejsu webowego bezpośrednio 
  na mikrokontrolerze oraz zapewnia komunikację pomiędzy ESP32 a interfejsem. 
  Wykorzystany w tym celu protokół WebSocket umożliwia natychmiastową synchronizację zmian między stroną internetową 
  a wyświetlaczem, co pozwala użytkownikowi na intuicyjny i interaktywny monitoring oraz kontrolę stanu Maszyny W.

\section{Analiza rozwiązań literaturowych}
