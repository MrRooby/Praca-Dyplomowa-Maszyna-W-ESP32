\chapter{Wstęp}
\label{ch:wstep}

\section{Cel pracy}
  Celem niniejszej pracy jest opracowanie systemu wizualizacji danych pochodzących z sieciowego interfejsu 
  Maszyny W z wykorzystaniem mikrokontrolera ESP32.

  Opracowane w ramach pracy dyplomowej urządzenie umożliwia odwzorowanie, w czasie rzeczywistym, wartości 
  liczbowych oraz sygnałów sterujących symulatorem Maszyny W na adresowalnych diodach LED 
  (z ang. \textit{Light-Emitting Diode}) typu RGB (z ang. \textit{Red, Green, Blue}).

  Urządzenie pozwala na zdalne sterowanie funkcjami symulatora zarówno za pomocą przycisków 
  fizycznych, jak i interfejsu internetowego. System obsługuje udostępnianie strony internetowej bezpośrednio 
  na mikrokontrolerze oraz zapewnia komunikację pomiędzy stroną a urządzeniem. 
  Wykorzystane rozwiązania umożliwiają natychmiastową synchronizację zmian między stroną internetową 
  a modelem Maszyny, co pozwala użytkownikowi na intuicyjny i interaktywny monitoring oraz kontrolę stanu Maszyny W.

% \clearpage
\section{Wprowadzenie do Maszyny W}
  Współczesne systemy komputerowe charakteryzują się wysokim stopniem złożoności zarówno pod względem architektury 
  sprzętowej, jak i warstwy programowej. Pomimo tego, podstawowe zasady ich działania pozostają zgodne z klasycznymi 
  założeniami architektury von Neumanna, sformułowanymi w połowie XX wieku \cite{bib:von-neumann}. 

  Maszyna W została zaprojektowana w latach siedemdziesiątych XX wieku na Politechnice Śląskiej przez zespół kierowany 
  przez prof. zw. dr inż. Stefana Węgrzyna jako uproszczony model komputera, przeznaczony do nauki podstaw architektury 
  komputerów oraz zasad działania procesora \cite{bib:minut-tutaj-momot}.

  Konstrukcja stanowi uproszczenie rzeczywistych jednostek obliczeniowych, przy jednoczesnym zachowaniu 
  kluczowych cech architektury von Neumanna, takich jak wspólna pamięć dla danych i rozkazów oraz sekwencyjne 
  wykonywanie instrukcji programu.

  % \clearpage
  Zgodnie z przyjętymi założeniami, Maszyna W składa się z trzech podstawowych bloków funkcjonalnych: pamięci operacyjnej,
  jednostki arytmetyczno-logicznej oraz układu sterującego. 
  W skład Maszyny W wchodzą następujące rejestry:
  \begin{itemize}
    \item \textbf{rejestr L} \textemdash{} 5-bitowy licznik rozkazów;
    \item \textbf{rejestr I} \textemdash{} 8-bitowy rejestr instrukcji. 5 mniej znaczących bitów stanowi rejestr AD,
          przechowujący adres argumentu rozkazu. 3 bardziej znaczące bity przechowują rejestr kodu rozkazu;
    \item \textbf{rejestr A} \textemdash{} 5-bitowy rejestr adresowy, służący do adresowania komórki pamięci,
          do której wykonywane są operacje zapisu i odczytu;
    \item \textbf{rejestr S} \textemdash{} 8-bitowy rejestr słowa, do którego zapisywane/odczytywane są wartości z pamięci;
    \item \textbf{rejestr AK} \textemdash{} 8-bitowy akumulator, do którego zapisywany jest wynik operacji arytmetycznej/logicznej.
  \end{itemize}

  Oraz następujące magistrale:
  \begin{itemize}
    \item \textbf{magistrala S} \textemdash{} 8-bitowa magistrala słowa,
    \item \textbf{magistrala A} \textemdash{} 5-bitowa magistrala adresowa.
  \end{itemize}
  
  Wymienione elementy znajdują się na poniższym schemacie urządzenia \ref{fig:rysunek-maszyny}.
  \begin{figure}[H]
    \centering
    \includegraphics[width=1\textwidth]{./graf/rysunek-maszyny.pdf}
    \caption{Schemat Maszyny W. Źródło: opracowanie własne na podstawie \cite{bib:cwiczenia-tutaj}.}
    \label{fig:rysunek-maszyny}
  \end{figure}
  
  \clearpage
  Pamięć operacyjna przechowuje zarówno dane, jak i rozkazy
  programu w formie 8-bitowego słowa. Jednostka arytmetyczno-logiczna (JAL) realizuje operacje arytmetyczne i logiczne 
  (m.in. dodawanie, odejmowanie), wykorzystując akumulator jako główny rejestr roboczy, natomiast układ sterujący odpowiada za pobieranie, dekodowanie 
  oraz wykonywanie rozkazów \cite{bib:cwiczenia-tutaj}.
  \\
  
  Istotnym elementem Maszyny W jest sposób sterowania przepływem danych pomiędzy jej komponentami. Komunikacja 
  realizowana jest za pomocą magistrali adresowej i magistrali słowa, przebieg operacji kontrolowany jest przez 
  zestaw sygnałów mikrosterujących (np. \textit{czyt, wys, wei, il}). Sygnały te, umożliwiają precyzyjne sterowanie 
  przesyłami międzypamięciowymi i międzypamięciowo-rejestrowymi w kolejnych taktach zegara \cite{bib:minut-tutaj-momot}. 
  Dzięki temu możliwe jest szczegółowe śledzenie procesu realizacji pojedynczego rozkazu na poziomie mikrooperacji.

  Z perspektywy dydaktycznej szczególnie istotna jest możliwość analizy pełnego cyklu rozkazu, obejmującego fazy 
  pobrania, dekodowania oraz wykonania instrukcji. W Maszynie W faza pobrania i dekodowania rozkazu jest identyczna 
  dla wszystkich instrukcji, co pozwala na wyraźne oddzielenie części wspólnej cyklu rozkazu od operacji 
  charakterystycznych dla konkretnego typu instrukcji. Takie podejście sprzyja lepszemu zrozumieniu mechanizmów 
  sterowania procesorem oraz zależności pomiędzy rejestrami i pamięcią.
  \\

  Maszyna W wykorzystywana jest obecnie głównie w postaci programowego symulatora, który umożliwia implementację i 
  uruchamianie własnych rozkazów asemblerowych. W ramach zajęć laboratoryjnych studenci projektują rozkazy, definiując 
  sekwencje mikrosygnałów realizujących zadane operacje \cite{bib:minut-brzeski}. Proces ten wymaga nie tylko znajomości
  struktury Maszyny W, lecz także umiejętności analizy algorytmicznej oraz świadomego zarządzania stanem rejestrów i 
  pamięci.
