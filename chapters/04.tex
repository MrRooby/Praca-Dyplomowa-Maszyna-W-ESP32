\chapter{Schematy bloków systemu wizualizacji}
\label{ch:04}
Poniższy rozdział ma na celu prezentację schematów elektronicznych dla systemu wizualizacji stanu interfejsu Maszyny W.
Schematy zostały wykonane przy użyciu otwartego oraz darmowego programu \textit{KiCAD}. Rysunki zostały sformatowane w widoczny 
sposób z uwagi na ich znaczący rozmiar. Rysunki zostały przygotowane w formacie papieru A3.  

\clearpage
\section{Płyta główna SWIM}
  Na rysunku (Rys. \ref{fig:plyta-glowna}) przedstawiono schemat płyty głównej systemu 
  wizualizacji interfejsu Maszyny W.

  \begin{figure}[H]
    \centering
    \includegraphics[angle=90, width=0.9\textwidth]{./graf/plyta-glowna.pdf}
    \caption{Schemat płyty głównej z mikroprocesorem ESP32-S3}
    \label{fig:plyta-glowna}
  \end{figure}

\section{Wyświetlacz siedmiosegmentowy trzycyfrowy}
  Na rysunku (Rys. \ref{fig:trzyseg-wysw}) przedstawiono schemat schemat wyświetlacza trzycyfrowego skonstruowanego z 
  połączonych ze sobą w linii ciągłej diod LED.

  \begin{figure}[H]
    \centering
    \includegraphics[angle=90, width=0.9\textwidth]{./graf/trzyseg-wysw.pdf}
    \caption{Wyświetlacz siedmiosegmentowy trzycyfrowy}
    \label{fig:trzyseg-wysw}
  \end{figure}

\clearpage
\section{Wyświetlacz siedmiosegmentowy dwucyfrowy}
  Na rysunku (Rys. \ref{fig:dwuseg-wysw}) przedstawiono schemat wyświetlacza dwucyfrowego skonstruowanego z 
  połączonych ze sobą w linii ciągłej diod LED.

  \begin{figure}[H]
    \centering
    \includegraphics[angle=90, width=0.9\textwidth]{./graf/dwuseg-wysw.pdf}
    \caption{Wyświetlacz siedmiosegmentowy trzycyfrowy}
    \label{fig:dwuseg-wysw}
  \end{figure}

\section{Połączenia pomiędzy blokami}
  TODO !!!