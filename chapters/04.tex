\chapter{Schematy bloków systemu wizualizacji}
\label{ch:04}

\section{System mikroprocesorowy – płyta główna}
Na rysunku (Rys. \ref{fig:plyta-glowna}) przedstawiono schemat systemu mikroprocesorowego.

\begin{figure}[H]
\centering
\includegraphics[width=1\textwidth]{./graf/plyta-glowna.pdf}
\caption{Schemat płyty głównej z mikroprocesorem ESP32-S3}
\label{fig:plyta-glowna}
\end{figure}

\section{Wyświetlacz siedmiosegmentowy trzycyfrowy}
Na rysunku (Rys. \ref{fig:trzyseg-wysw}) przedstawiono schemat schemat wyświetlacza trzycyfrowego skonstruowanego z 
połączonych ze sobą w linii ciągłej diod LED.

\begin{figure}[H]
\centering
\includegraphics[width=1\textwidth]{./graf/trzyseg-wysw.pdf}
\caption{Wyświetlacz siedmiosegmentowy trzycyfrowy}
\label{fig:trzyseg-wysw}
\end{figure}

\clearpage
\section{Wyświetlacz siedmiosegmentowy dwucyfrowy}
Na rysunku (Rys. \ref{fig:dwuseg-wysw}) przedstawiono schemat wyświetlacza dwucyfrowego skonstruowanego z 
połączonych ze sobą w linii ciągłej diod LED.

\begin{figure}[H]
\centering
\includegraphics[width=1\textwidth]{./graf/dwuseg-wysw.pdf}
\caption{Wyświetlacz siedmiosegmentowy trzycyfrowy}
\label{fig:dwuseg-wysw}
\end{figure}

\section{Połączenia pomiędzy blokami}
TODO totalnie nie wiem co tutaj napisać do prowadzącego