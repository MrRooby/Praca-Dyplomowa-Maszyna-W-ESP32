\chapter{Część projektowa}
% \label{ch:wymagania-i-narzedzia} % nie wiem do czego to służy więc nie usuwam póki co

\section{System mikrokontrolerowy}
\label{sec:esp32}
  Ze względu na wymaganą komunikację sieciową projektu, wysoką złożoność rozwiązania oraz zapotrzebowanie 
  na wiele portów wejścia/wyjścia należało wybrać system mikrokontrolerowy zapewniający poniższą funkcjonalność:
  \begin{itemize}
    \item obsługa dwóch kanałów dla adresowalnych diod LED RGB,
    \item obsługa 17 przycisków wraz z enkoderem,
    \item zapewnienie punktu dostępu do internetu,
    \item przechowywanie danych w pamięci nieulotnej,
    \item udostępnianie strony internetowej.
  \end{itemize}
  
  Biorąc pod uwagę powyższe wymagania, rozważono dwie możliwości: mikrokomputer z rodziny \textit{Raspberry Pi} lub 
  mikrokontroler z rodziny ESP32. Zarówno mikrokomputer jak i mikrokontroler spełniają wszystkie ww. założenia. Urządzenia 
  z rodziny \textit{Raspberry Pi} charakteryzują się prostotą obsługi oraz konfiguracji, są wyposażone w system operacyjny oparty na 
  jądrze \textit{Linuxa} oraz pozwalają na programowanie w językach \textit{Python} oraz \textit{C++}. System operacyjny pozwala 
  również na łatwe przechowywanie danych w pamięci nieulotnej urządzenia z uwagi na obecny w nim system plików. 
  Obecny na urządzeniach \textit{Raspberry Pi} system operacyjny \textit{Raspbian} będący pochodną popularnej dystrybucji 
  \textit{Debiana} pozwala na tworzenie własnych procesów typu \textit{daemon} (procesów działających w tle). 
  

  Mikrokontrolery z rodziny ESP32 nie posiadają systemu operacyjnego, a ich programowanie jest możliwe przy użyciu języków 
  takich jak \textit{C} oraz \textit{C++}. Służą do tego specjalnie przygotowane środowiska programistyczne takie jak 
  \textit{Visual Studio Code} z rozszerzeniem \textit{PlatformIO} a także \textit{Arduino IDE}. Ich przewagą nad mikrokomputerami 
  są: brak systemu operacyjnego, niższa cena, mniejsza złożoność oraz mniejsze zużycie energii. 
  
  \clearpage
  Po rozważeniu obu opcji zdecydowano się na użycie mikrokontrolera z rodziny ESP32. Wybór został podyktowany pozwalającym na pełną 
  kontrolę urządzenia językiem \textit{C++}, niższą ceną, brakiem systemu operacyjnego, dostępnością oraz bogatą dokumentacją. 
  Z katalogu firmy \textit{Espressif Systems} wybrano urządzenie \textit{ESP32-S3-WROOM-1}.

  Poniższy rysunek (Rys. \ref{fig:schemat-blokowy-esp32}) z dokumentacji producenta przedstawia schemat funkcjonalny wybranego układu.
  \begin{figure}[H]
  \centering
  \includegraphics[width=0.7\textwidth]{./graf/schemat-esp32.png}
    \caption{Schemat blokowy funkcjonalny układu ESP32-S3. 
    Źródło: \url{https://www.circuitstate.com/wp-content/uploads/2023/08/Espressif-ESP32-S3-Functional-Block-Diagram-CIRCUITSTATE-Electronics-1.png}}
  \label{fig:schemat-blokowy-esp32}
  \end{figure}


  Własności układu zapożyczone z dokumentacji firmy \textit{Espressif Systems} \cite{bib:esp-doku}:
  \begin{itemize}
    \item dwurdzeniowy 32-bitowy mikrokontroler \textit{Xtensa LX7},
    \item 16~MB pamięci flash, 16MB pamięci PSRAM,
    \item 512~KB pamięci SRAM,
    \item 36 portów GPIO,
    \item komunikacja 2.4~GHz WiFi (802.11b/g/n) oraz Bluetooth 5,
    \item wbudowana antena,
    \item wsparcie dla: I2C, LED PWM, USB Serial,
    \item maksymalny pobór prądu ~300-400~mA,
    \item zasilanie napięciem 3.3~V.
  \end{itemize}

\clearpage
\section{Adresowalne diody LED}
\label{sec:led-rgb}
  Przyjęto założenie dotyczące wizualizacji interfejsu Maszyny W na tablicy złożonej z diod LED RGB. Adresowalne diody LED RGB pozwalają
  na sekwencyjne zadanie kolorów każdej z 3 diod (czerwonej, zielonej oraz niebieskiej) elementu, połączonych w ciągłą linię.
  Przewidywana liczba pojedynczych diod LED podłączonych do SWIM przekracza 600. Znaczna liczba elementów może doprowadzać do
  wpływających na poprawne działanie układu spadków napięcia.

  Spośród aktualnie dostępnych na rynku wersji elementu wyróżniły się dwa modele: \textit{WS2812B} oraz \textit{WS2815B} firmy
  \textit{WORLDSEMI}.
  Podczas wstępnych testów obu elementów zaobserwowano następujące cechy. Diody \textit{WS2812B} zasilane napięciem 5~V wykazywały
  zauważalny spadek jakości odwzorowania kolorów przy liczbie elementów przekraczającej 100 (w ciągłej linii). Wykorzystanie 
  ww. elementów niosłoby za sobą potrzebę dodatkowego wprowadzania napięcia do linii diod. Diody \textit{WS2815B} zasilane napięciem 
  12~V nie wykazały zauważalnych spadków napięcia przy tej samej liczbie elementów. Cecha ta zadecydowała o ostatecznym wyborze 
  modelu \textit{WS2815B-V1}, jego jedyną wadą jest nieznacznie większa cena w porównaniu do 5~V odpowiednika. 
  \\


  Poniższy rysunek (Rys. \ref{fig:schemat-LED-RGB}) z dokumentacji producenta przedstawia schemat wyprowadzeń elementu \cite{bib:led-doku}.
  \begin{figure}[H]
  \centering
  \includegraphics[width=0.4\textwidth]{./graf/led-rgb-schematic.png}
  \caption{Schemat wyprowadzeń adresowalnej diody LED RGB \textit{WS2815B-V1}.
    Źródło: \url{https://www.alldatasheet.com/htmldatasheet2/1134588/WORLDSEMI/WS2815B/1139/2/WS2815B.png}}
  \label{fig:schemat-LED-RGB}
  \end{figure}

  Własności elementu:
  \begin{itemize}
    \item napięcie pracy 12~V,
    \item pobór prądu 15~mA przy maksymalnej jasności każdej z 3 kolorowych diod,
    \item montaż SMT (z ang. \textit{surface-mount technology}),
    \item częstotliwość odświeżania 4~kHz.
  \end{itemize}

\clearpage
\section{Przyciski}
\label{sec:przyciski}
  W celu zapewnienia pełnej obsługi Maszyny W potrzebne jest wykorzystanie 17 przycisków.
  Symulator w wersji podstawowej udostępnia użytkownikowi 16 możliwych sygnałów: 
  \textit{czyt, wys, wei, il, wyad, weak, pisz, przep, wel, wyl, dod, ode, wes, weja, wyak, wea} \cite{bib:cwiczenia-tutaj}. Do wymienionych sygnałów 
  należy również wliczyć przycisk \textit{takt}, który zapewni możliwość wykonania zadanego rozkazu.

  Zdecydowano się na wykorzystanie przełączników \textit{MX Blue} firmy \textit{Cherry}. Według dokumentacji producenta
  wyżej wymieniony przełącznik klawiaturowy pozwala na 50 milionów aktywacji \cite{bib:cherry}. 
  Pozwoli to na długoletnią, niezawodną działalność.
  Wykorzystano przyciski typu \textit{Blue} oferujące słyszalny odgłos kliknięcia przy wciskaniu. Daje to użytkownikowi
  informację zwrotną o stanie przełącznika.

  Aby ograniczyć liczbę potrzebnych wyprowadzeń dla przycisków zdecydowano wykorzystać multiplekser. 
  W tym celu wybrano 16 kanałowy multiplekser z wyprowadzeniem \textit{COM}. Urządzenie to pozwala na odczyt stanu przycisku wprowadzając 
  na wejście multipleksera 4-bitowego sygnału. Cykliczny odczyt stanu wyprowadzenia \textit{COM} wraz z odpowiednią kombinacją 
  4-bitowego wejścia pozwala na odczyt aktualnego stanu przycisków. Zmiana kombinacji 4-bitów wejścia oraz odczyt stanu wyprowadzenia
  \textit{COM} co cykl mikrokontrolera pozwala na odczyt aktualnego stanu przycisków. Z uwagi na niską cenę oraz niewielki rozmiar 
  wybrano model \textit{CD74HC4067M} firmy \textit{Texas Instruments}.
  
  Poniższy rysunek (Rys. \ref{fig:schemat-mux}) z dokumentacji producenta przedstawia diagram funkcjonalny układu \cite{bib:mux-doku}.
  \begin{figure}[H]
    \centering
    \includegraphics[width=0.7\textwidth]{./graf/mux-schematic.png}
    \caption{Diagram funkcjonalny multipleksera \textit{CD74HC4067M}. 
    Źródło: \url{https://www.alldatasheet.com/html-pdf/517535/TI/CD74HC4067M/62/2/CD74HC4067M.html}}
    \label{fig:schemat-mux}
  \end{figure}
  
  Poniższa tabela (Tab. \ref{tab:mux-truth-table}) prawdy przedstawia odczyt stanu przycisków.
  \begin{table}[H]
    \centering
    \begin{tabular}{|c|c|c|c|c|c|}
      \hline
      \textbf{S0} & \textbf{S1} & \textbf{S2} & \textbf{S3} & \textbf{COM} & \textbf{Kanał} \\ \hline
      0  & 0 & 0 & 0 & 1 & 0 \\ \hline
      1  & 0 & 0 & 0 & 1 & 1 \\ \hline
      0  & 1 & 0 & 0 & 1 & 2 \\ \hline
      1  & 1 & 0 & 0 & 1 & 3 \\ \hline
      0  & 0 & 1 & 0 & 1 & 4 \\ \hline
      1  & 0 & 1 & 0 & 1 & 5 \\ \hline
      0  & 1 & 1 & 0 & 1 & 6 \\ \hline
      1  & 1 & 1 & 0 & 1 & 7 \\ \hline
      0  & 0 & 0 & 1 & 1 & 8 \\ \hline
      1  & 0 & 0 & 1 & 1 & 9 \\ \hline
      0  & 1 & 0 & 1 & 1 & 10 \\ \hline
      1  & 1 & 0 & 1 & 1 & 11 \\ \hline
      0  & 0 & 1 & 1 & 1 & 12 \\ \hline
      1  & 0 & 1 & 1 & 1 & 13 \\ \hline
      0  & 1 & 1 & 1 & 1 & 14 \\ \hline
      1  & 1 & 1 & 1 & 1 & 15 \\ \hline
    \end{tabular}
    \caption{Tablica prawdy multipleksera}
    \label{tab:mux-truth-table}
  \end{table}

\clearpage
\section{Podświetlenie LED}
  Chcąc zapewnić przyjazny wizualnie wygląd modelu SWIM zdecydowano się na podświetlenie następujących elementów:
  \begin{itemize}
    \item licznika rozkazów,
    \item akumulatora,
    \item rejestru instrukcji,
    \item rejestrów A oraz S,
    \item loga Politechniki Śląskiej.
  \end{itemize}

  Wykorzystano do tego 12~V linie diod LED.
  Aby uniknąć zauważalnych dla użytkownika spadków jasności podświetlenia zdecydowano na podłączenie 
  bezpośrednio do zasilacza 12~V.

\section{Enkoder}
  W celu zapewnienia możliwości wprowadzania wartości do: licznika rozkazów, akumulatora oraz rejestrów w lokalnym trybie
  działania urządzenia zastosowano enkoder obrotowy z przyciskiem.

  Zamierzone działanie enkodera:
  \begin{itemize}
    \item zwiększanie/zmniejszanie wartości liczbowych w zależności od kierunku obrotu;
    \item aktywacja trybu wprowadzania danych poprzez przytrzymanie przycisku;
    \item przełączanie między polami wprowadzania danych przy użyciu szybkich kliknięć.
  \end{itemize}

  Enkoder pozwala użytkownikowi w prosty i intuicyjny sposób sterować parametrami urządzenia bez korzystania z 
  interfejsu sieciowego. Cykliczny odczyt stanów dwóch kanałów enkodera umożliwiał stwierdzenie strony rotacji.

  \clearpage
  Aby określić kierunek obrotu elementu należy odczytać zmianę stanów na wyprowadzeniach urządzenia. Poniższe wykresy 
  przedstawiają sposób odczytu obrotu w prawo (Rys. \ref{fig:enc-cw}, Tab. \ref{tab:enc-cw}) oraz w lewo 
  (Rys. \ref{fig:enc-ccw}, Tab. \ref{tab:enc-ccw}).

\begin{figure}[H]
  \centering
  % --- Pierwsza sekcja (Obrót w prawo) ---
  \begin{minipage}[t]{0.58\textwidth} % Zwiększona szerokość dla rysunku
    \centering
    \vfill
    \includegraphics[width=\textwidth]{./graf/enc-graph-cw.pdf}
    \caption{Wykres stanów przy obrocie w prawo. Źródło: opracowanie własne.}
    \label{fig:enc-cw}
  \end{minipage}
  \hfill
  \begin{minipage}[t]{0.38\textwidth} % Zmniejszona szerokość dla tabeli
    \centering
    \vspace{0pt} % Trick, aby góra tabeli zrównała się z górą rysunku
    \begin{tabular}{|c|c|c|}
      \hline
      \textbf{Krok} & \textbf{A} & \textbf{B} \\ \hline
      1 & 0 & 1 \\ \hline
      2 & 0 & 0 \\ \hline
      3 & 1 & 0 \\ \hline
      4 & 1 & 1 \\ \hline
    \end{tabular}
    \captionof{table}{Tablica prawdy dla obrotu w prawo}
    \label{tab:enc-cw}
  \end{minipage}

  \vspace{0.8cm}

  % --- Druga sekcja (Obrót w lewo) ---
  \begin{minipage}[t]{0.58\textwidth}
    \centering
    \vfill
    \includegraphics[width=\textwidth]{./graf/enc-graph-ccw.pdf}
    \caption{Wykres stanów przy obrocie w lewo. Źródło: opracowanie własne.}
    \label{fig:enc-ccw}
  \end{minipage}
  \hfill
  \begin{minipage}[t]{0.38\textwidth}
    \centering
    \vspace{0pt}
    \begin{tabular}{|c|c|c|}
      \hline
      \textbf{Krok} & \textbf{A} & \textbf{B} \\ \hline
      1 & 1 & 0 \\ \hline
      2 & 0 & 0 \\ \hline
      3 & 0 & 1 \\ \hline
      4 & 1 & 1 \\ \hline
    \end{tabular}
    \captionof{table}{Tablica prawdy dla obrotu w lewo}
    \label{tab:enc-ccw}
  \end{minipage}
\end{figure}

\section{Przełącznik trybów}
\label{sec:przel_tryb}
  Jednym z dodatkowych założeń projektu jest możliwość przełączania SWIM między trybami lokalnym oraz sieciowym. 
  Aby zapewnić taką możliwość bezpośrednio z poziomu modelu wykorzystano przełącznik bistabilny. Podłączenie 
  przełącznika bezpośrednio do systemu mikrokontrolerowego pozwala na wykorzystanie przerwań w celu natychmiastowej
  zmiany trybu pracy urządzenia. Korekta drgań styków urządzenia została zapewniona z poziomu oprogramowania. 

\clearpage
\section{Serwer} %żródło kto zrobił stronę
\label{sec:serwer}
  Na potrzebę modernizacji narzędzia dydaktycznego, jakim jest Maszyna W, została stworzona strona 
  internetowa symulująca działanie modelu komputera \cite{bib:aei-maszynaw}. Projekt oraz wykonanie ww. strony 
  internetowej nie jest częścią bieżącej pracy dyplomowej. 
  Zadaniem SWIM jest udostępnianie ww. strony internetowej na lokalnej sieci utworzonej przez mikrokontroler.


  Mikrokontroler wybrany w rozdziale \ref{sec:esp32} umożliwia udostępnianie złożonych stron internetowych z poziomu urządzenia.
  Aby było to możliwe planowano wykorzystać biblioteki takie jak:
  \begin{itemize}
    \item \textit{WiFi} - obsługa połączeń i konfiguracji interfejsu sieci bezprzewodowej,
    \item \textit{AsyncTCP} - asynchroniczna, nieblokująca komunikacja TCP 
      (z ang. \textit{Transmission Control Protocol}),
    \item \textit{AsyncWebServer} - asynchroniczna obsługa serwera HTTP 
      (z ang. \textit{Hypertext Transfer Protocol}),
    \item \textit{LittleFS} - system plików dla pamięci flash mikrokontrolera,
    \item \textit{DNSServer} - lokalna obsługa i przekierowywanie zapytań DNS 
      (z ang. \textit{Domain Name Server}).
  \end{itemize} 


  Wykorzystanie owych bibliotek pozwala na:
  \begin{itemize}
    \item kontrolę połączeń z użytkownikami oraz ograniczenie ich liczby,
    \item komunikację przy wykorzystaniu protokołu sieciowego TCP,
    \item przechowywanie złożonej strony internetowej Maszyny W bezpośrednio w pamięci flash mikrokontrolera,
    \item wykorzystanie portalu przechwytującego zapytania DNS w celu przekierowania użytkowników na wybraną stronę.
  \end{itemize} 

\section{Punkt dostępowy}
  Wykorzystanie urządzenia sieciowego wymaga zabezpieczenia rozwiązania przed niechcianym dostępem oraz atakami.
  Spośród dwóch możliwych rozwiązań \textemdash{} udostępniania strony internetowej w sieci globalnej bądź lokalnej \textemdash{} wybrano rozwiązanie w pełni lokalne.
  Aby było to możliwe mikrokontroler tworzy własną sieć internetową na której następnie serwuje stronę internetową.  
  

  Tworzenie własnego punktu dostępu oraz izolacja go od sieci globalnej zapewnia bezpieczeństwo oraz kontrolę dostępu dla urządzenia.
  Bibioteki wymienione w rozdziale \ref{sec:serwer} umożliwiają zadanie hasła punktowi dostępu oraz ograniczenie liczby połączonych na raz użytkowników.

\section{Strona internetowa}
  Bieżąca praca dyplomowa wykorzystuje wykonaną na potrzeby projektu PBL 
  (z ang. \textit{Project-Based Learning}), którego SWIM również był częścią, stronę 
  modernizującą symulator Maszyny W. Niezwykle ważnym jest zaznaczyć, iż projekt oraz wykonanie ww. strony internetowej
  nie są częścią bieżącej pracy dyplomowej. Została ona stworzona przez członków projektu naukowego zajmującego się 
  modernizacją Maszyny W \cite{bib:aei-maszynaw}. Mając na uwadze autorów tej części projektu należy zrozumieć jej działanie,
  aby w odpowiedni sposób wdrożyć ją do projektu. 
  \\

  Strona umożliwia naukę działania Maszyny W w intuicyjny oraz łatwo dostępny sposób. Na Rys. \ref{fig:maszyna-web} widnieje 
  strona główna symulatora.
  \begin{figure}[H]
  \centering
  \includegraphics[width=1\textwidth]{./graf/maszyna-web.png}
    \caption{Strona główna sieciowego symulatora Maszyny W. Źródło: \url{https://maszynaw.aei.polsl.pl}}
  \label{fig:maszyna-web}
  \end{figure}

  Zakłada się wizualizację obecnych na powyższym rysunku elementów symulatora, takich jak: licznik, akumulator, czy pamięć operacyjna
  przy pomocy wymienionych w rodziale \ref{sec:led-rgb} wyświetlaczy. Kontrolę sygnałów takich jak: czyt, wys, wei, itd. umożliwią 
  opisane w rozdziale \ref{sec:przyciski} przyciski. 

\clearpage
\section{Interfejs komunikacyjny}
  Niezbędnym elementem bieżącego projektu jest możliwość komunikacji pomiędzy mikrokontrolerem, a udostępnianą na nim stroną w czasie rzeczywistym.
  Wyjściowo, nie jest możliwe przesyłanie informacji o aktualnie znajdujących się wartościach w polach tekstowych strony internetowej lub o
  aktualnie wybranych sygnałach. 
  
  Standardowa implementacja serwera posiada jedynie możliwość komunikacji za pomocą protokołu HTTP. Wykorzystanie
  takiego rozwiązania wymagałoby wysyłania cyklicznych żądań do serwera co powodowałoby wysokie zużycie zasobów oraz opóźnienia w działaniu.  
  W celu zapewnienia wydajnej oraz szybkiej komunikacji pomiędzy stroną internetową a serwerem wykorzystano technologię \textit{WebSocket}.
  \\

  \textit{Websocket} jest dwukierunkowym protokołem komunikacyjnym. Działa zarówno w warstwie aplikacji (strona internetowa) oraz 
  na serwerze (mikrokontroler). Umożliwia wymianę danych w czasie rzeczywistym, bez konieczności każdorazowego inicjowania żądania przez użytkownika.
  Wykorzystuje asynchroniczną komunikację, co pozwala na nieblokujące odbieranie i nadawanie złożonych informacji \cite{bib:websocket}.


  Poniższy schemat ilustruje przebieg komunikacji \textit{WebSocket}, pomiędzy klientem (stroną internetową), a serwerem.
  \begin{figure}[H]
    \centering
    \includegraphics[width=0.8\textwidth]{./graf/websocket-graph.pdf}
    \caption{Schemat komunikacji przy wykorzystaniu protokołu \textit{WebSocket}. Źródło: opracowanie własne.}
    \label{fig:websocket-graph}
  \end{figure} 

\clearpage
\section{Zasilanie}
  Mając na uwadze wszystkie wymienione w powyższym rozdziale elementy SWIM, dokonano analizy wymagań energetycznych układu w celu dobrania 
  odpowiedniego zasilacza. Adresowalne diody LED oraz podświetlenie wymagały zasilania napięciem 12~V, natomiast mikrokontroler ESP32 
  oraz multiplekser korzystały z napięcia 3,3~V.
  \\

  Przeprowadzono szacunkowe obliczenia maksymalnego poboru prądu:
  \begin{itemize}
    \item dla linii diod LED \textit{WS2815B-V1}, przy założeniu pełnej jasności każdej z trzech diod RGB oraz wykorzystaniu 600 elementów, przewidziano maksymalny pobór prądu na poziomie około 9~A,
    \item dla podświetlenia LED przewidziano pobór prądu rzędu 0,5~A, uwzględniając wszystkie elementy podświetlane jednocześnie,
    \item dla mikrokontrolera ESP32-S3 oraz układów towarzyszących przewidziano pobór prądu około 600~mA przy pełnym obciążeniu interfejsów peryferyjnych i aktywnym WiFi.
  \end{itemize}

Na podstawie powyższych obliczeń dobrano zasilacz impulsowy o napięciu wyjściowym 12~V i mocy minimalnej 330~W, co zapewniało bezpieczny 
margines względem przewidywanego maksymalnego poboru prądu. Zastosowanie zasilacza o większej mocy umożliwiało również stabilne działanie 
układu przy chwilowych skokach poboru prądu przez linie LED.

Zasilanie mikrokontrolera ESP32 oraz innych układów niskonapięciowych realizowano przy użyciu przetwornicy DC-DC obniżającej napięcie z 12~V do 3,3~V.
Rozwiązanie to zapewniało izolację układów niskonapięciowych od wysokoprądowych linii LED oraz stabilne napięcie zasilania, eliminując spadki i 
zakłócenia w pracy mikrokontrolera. Dodatkowo, w celu ochrony układu przed odwrotną polaryzacją, należy zapewnić odpowiednią ochronę. 
Takie podejście zapewniło bezpieczne i niezawodne działanie całego systemu SWIM.
