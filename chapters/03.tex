\chapter{Część projektowa}
% \label{ch:wymagania-i-narzedzia} % nie wiem do czego to służy więc nie usuwam póki co

\section{System mikroprocesorowy}
  Ze względu na wymaganą komunikację sieciową projektu, wysoką złożoność rozwiązania oraz zapotrzebowanie na wiele portów wejścia/wyjścia
  należało wybrać system mikroprocesorowy zapewniający poniższe możliwości:
  \begin{itemize}
    \item obsługa dwóch kanałów dla adresowalnych diod LED RGB,
    \item obsługa 17 przycisków wraz z enkoderem,
    \item zapewnienie punktu dostępu internetu,
    \item przechowywanie danych w pamięci nieulotnej,
    \item serwowanie strony internetowej.
  \end{itemize}
  
  Zwracając uwagę na powyżej wymienione wymagania rozważono dwie możliwości: mikrokomputer z rodziny \textit{Raspberry Pi} lub 
  mikrokontroler z rodziny \textit{ESP32}. Zarówno mikrokomputer jak i mikrokontroler spełniają wszystkie ww. założenia. Urządzenia 
  z rodziny \textit{Raspberry Pi} są proste w użytku, posiadają system operacyjny oparty na jądrze \textit{Linux'a} oraz pozwalają na 
  programowanie w językach \textit{Python} oraz \textit{C++}. System operacyjny pozwala również na łatwe przechowywanie danych
  w pamięci nieulotnej urządzenia z uwagi na obecny w nim system plików. Obecny na urządzeniach \textit{Raspberry Pi} system operacyjny
  \textit{Raspbian} będący pochodną popularnej dystrybucji \textit{Debiana} pozwala na tworzenie własnych \textit{daemonów}
  (procesy działające w tle). 
  

  Mikrokontrolery z rodziny \textit{ESP32} nie posiadają systemu operacyjnego, a ich programowanie jest możliwe przy użyciu języków takich jak \textit{C} oraz 
  \textit{C++}. Służą do tego specjalnie przygotowane środowiska programistyczne takie jak \textit{Visual Studio Code} z rozszerzeniem \textit{PlatformIO} a także
  \textit{Arduino IDE}. Ich przewagą nad mikrokomputerami są: brak systemu operacyjnego, niższa cena, mniejsza złożoność oraz mniejsze zużycie energii. 
  
  \clearpage
  Po rozważeniu obu opcji decydowano się na użycie mikrokontrolera z rodziny ESP32. Wybór został podyktowany pozwalającym na pełną kontrolę urządzenia
  językiem \textit{C++}, niższą ceną, brakiem systemu operacyjnego, dostępnością oraz bogatą dokumentacją. Z katalogu firmy \textit{Espressif Systems} 
  wybrano urządzenie \textit{ESP32-S3-WROOM-1}.

  Poniższy rysunek (Rys. \ref{fig:schemat-blokowy-esp32}) z dokumentacji producenta przedstawia schemat funkcjonalny wybranego układu
  \begin{figure}[H]
  \centering
  \includegraphics[width=0.8\textwidth]{./graf/schemat-esp32.png}
  \caption{Schemat blokowy funkcjonalny układu ESP32-S3}
  \label{fig:schemat-blokowy-esp32}
  \end{figure}


  Własności wybranego układu zapożyczone z dokumentacji firmy \textit{Espressif Systems}:
  \begin{itemize}
    \item dwurdzeniowy 32-bitowy mikroprocesor \textit{Xtensa LX7},
    \item 16MB pamięci flash, 16MB pamięci PSRAM,
    \item 512MB pamięci SRAM,
    \item 36 portów GPIO,
    \item komunikacja 2.4GHz WiFi (802.11b/g/n) oraz Bluetooth 5,
    \item wbudowana antena,
    \item wsparcie dla: I2C, LED PWM, USB Serial,
    \item maksymalny pobór prądu ~300-400mA,
    \item zasilanie napięciem 3.3V.
  \end{itemize}

\clearpage
\section{Adresowalne diody LED}
  Zakładano wizualizację interfejsu Maszyny W na tablicy złożonej z diod LED RGB. Adresowalne diody LED RGB pozwalają
  na sekwencyjne zadanie kolorów każdej z 3 diod (czerwonej, zielonej oraz niebieskiej) elementu, połączonych w ciągłą linię.
  Przewidywana liczba pojedynczych diod LED podłączonych do SWIM przekracza 600. Znaczna liczba elementów może doprowadzać do
  wpływających na poprawne działanie układu spadków napięcia.

  Spośród aktualnie dostępnych na rynku wersji elementu wyróżniły się dwa modele: \textit{WS2812B} oraz \textit{WS2815B} firmy \textit{WORLDSEMI}.
  Podczas wstępnych testów obu elementów zaobserwowano następujące cechy. Diody \textit{WS2812B} zasilane napięciem 5V wykazywały
  zauważalny spadek jakości odwzorowania kolorów przy liczbie elementów przekraczającej 100 (w ciągłej linii). Wykorzystanie 
  ww. elementów niosłoby za sobą potrzebę dodatkowego wprowadzania napięcia do linii diod. Diody \textit{WS2815B} zasilane napięciem 
  12V nie wykazały zauważalnych spadków napięcia przy tej samej liczbie elementów. Cecha ta zadecydowała o ostatecznym wyborze 
  modelu \textit{WS2815B-V1}, jego jedyną wadą jest nieznacznie większa cena w porównaniu do 5V odpowiednika. 
  \\


  Poniższy rysunek (Rys. \ref{fig:schemat-LED-RGB}) z dokumentacji producenta przedstawia schemat wyprowadzeń elementu.
  \begin{figure}[H]
  \centering
  \includegraphics[width=0.6\textwidth]{./graf/led-rgb-schematic.png}
  \caption{Schemat wyprowadzeń adresowalnej diody LED RGB \textit{WS2815B-V1}}
  \label{fig:schemat-LED-RGB}
  \end{figure}

  Własności elementu:
  \begin{itemize}
    \item napięcie pracy 12V,
    \item pobór prądu 15mA przy maksymalnej jasności każdej z 3 kolorowych diod,
    \item montaż SMD,
    \item częstotliwość odświeżania 4kHz.
  \end{itemize}


\section{Przyciski}
  W celu zapewnienia pełnej obsługi Maszyny W potrzebne jest wykorzystanie 17 przycisków.
  Symulator w wersji podstawowej udostępnia użytkownikowi 16 możliwych sygnałów: 
  \textit{czyt, wys, wei, il, wyad, weak, pisz, przep, wel, wyl, dod, ode, wes, weja, wyak, wea}. Do wymienionych sygnałów 
  należy również wliczyć przycisk \textit{takt}, który zapewni możliwość wykonania zadanego rozkazu.

  Zdecydowano się na wykorzystanie przełączników \textit{MX Blue} firmy \textit{Cherry}. Według dokumentacji producenta
  wyżej wymieniony przełącznik klawiaturowy pozwala na 50 milionów aktywacji. Pozwoli to na długoletnią, niezawodną działalność.
  Wykorzystano przyciski typu \textit{Blue} oferujące słyszalny odgłos kliknięcia przy wciskaniu. Daje to użytkownikowi
  informację zwrotną o stanie przełącznika.

  Aby ograniczyć liczbę potrzebnych wyprowadzeń dla przycisków zdecydowano wykorzystać multiplekser. 
  W tym celu wybrano 16 kanałowy multiplekser z wyprowadzeniem \textit{COM}. Urządzenie to pozwala na odczyt stanu przycisku wprowadzając 
  na wejście multipleksera 4-bitowego sygnału. Cykliczny odczyt stanu wyprowadzenia \textit{COM} wraz z odpowiednią kombinacją 
  4-bitowego wejścia pozwala na odczyt aktualnego stanu przycisków. Zmiana kombinacji 4-bitów wejścia oraz odczyt stanu wyprowadzenia
  \textit{COM} co cykl mikroprocesora pozwala na odczyt aktualnego stanu przycisków. Z uwagi na niską cenę oraz niewielki wymiar 
  wybrano model \textit{CD74HC4067M} firmy \textit{Texas Instruments}.
  
  Poniższy rysunek (Rys. \ref{fig:schemat-mux}) z dokumentacji producenta przedstawia diagram funkcjonalny układu.
  \begin{figure}[H]
  \centering
  \includegraphics[width=0.7\textwidth]{./graf/mux-schematic.png}
  \caption{Diagram funkcjonalny multipleksera \textit{CD74HC4067M}}
  \label{fig:schemat-mux}
  \end{figure}
  
  Poniższa tabela (Tab. \ref{tab:mux-truth-table}) prawdy przedstawia odczyt stanu przycisków
  \begin{table}[H]
  \centering
  \caption{Tablica prawdy multipleksera}
  \label{tab:mux-truth-table}
  \begin{tabular}{|c|c|c|c|c|c|}
  \hline
  \textbf{S0} & \textbf{S1} & \textbf{S2} & \textbf{S3} & \textbf{COM} & \textbf{Kanał} \\ \hline
  0  & 0 & 0 & 0 & 1 & 0 \\ \hline
  1  & 0 & 0 & 0 & 1 & 1 \\ \hline
  0  & 1 & 0 & 0 & 1 & 2 \\ \hline
  1  & 1 & 0 & 0 & 1 & 3 \\ \hline
  0  & 0 & 1 & 0 & 1 & 4 \\ \hline
  1  & 0 & 1 & 0 & 1 & 5 \\ \hline
  0  & 1 & 1 & 0 & 1 & 6 \\ \hline
  1  & 1 & 1 & 0 & 1 & 7 \\ \hline
  0  & 0 & 0 & 1 & 1 & 8 \\ \hline
  1  & 0 & 0 & 1 & 1 & 9 \\ \hline
  0  & 1 & 0 & 1 & 1 & 10 \\ \hline
  1  & 1 & 0 & 1 & 1 & 11 \\ \hline
  0  & 0 & 1 & 1 & 1 & 12 \\ \hline
  1  & 0 & 1 & 1 & 1 & 13 \\ \hline
  0  & 1 & 1 & 1 & 1 & 14 \\ \hline
  1  & 1 & 1 & 1 & 1 & 15 \\ \hline
  \end{tabular}
  \end{table}

\clearpage
\section{Podświetlenie LED}
  Chcąc zapewnić przyjazny wizualnie wygląd modelu SWIM zdecydowano się na podświetlenie następujących elementów:
  \begin{itemize}
    \item licznik rozkazów,
    \item akumulator,
    \item rejestr instrukcji,
    \item rejestry A oraz S,
    \item logo Politechniki Śląskiej.
  \end{itemize}

  Wykorzystano do tego 12V linie diod LED.
  Aby uniknąć zauważalny dla użytkowników spadków jasności podświetlenia zdecydowano na podłączenie bezpośrednio do 
  zasilacza 12V.
  % TODO czy tutaj napisać że mam tranzysotry na płytce których jednak nie będę wykorzystywał bo ledy gasną jak się zapalają
  % wsy? Czy póżniej o tym wspomineć?

\section{Enkoder}
  W celu zapewnienia możliwości wprowadzania wartości do: licznika rozkazów, akumulatora oraz rejestrów w lokalnym trybie
  działania urządzenia zastosowano enkoder obrotowy z przyciskiem.

  Zamierzone działanie enkodera:
  \begin{itemize}
    \item zwiększanie/zmniejszanie wartości liczbowych w zależności od kierunku obrotu,
    \item aktywacja trybu wprowadzania danych poprzez przytrzymanie przycisku,
    \item przełączanie między polami wprowadzania danych przy użyciu szybkich kliknięć.
  \end{itemize}

  Enkoder pozwala użytkownikowi w prosty i intuicyjny sposób sterować parametrami urządzenia bez korzystania z 
  interfejsu sieciowego. Cykliczny odczyt stanów dwóch kanałów enkodera umożliwiał stwierdzenie strony rotacji.

  \clearpage
  Działanie elementu można przedstawić przy pomocy wykresów (Rys. \ref{fig:encoder-graph}) stanów wyprowadzeń jego dwóch kanałów
  \textit{A} oraz \textit{B}. %źródło https://www.allaboutcircuits.com/uploads/articles/rotary-encoder-waveform-v2.jpg
  \begin{figure}[H]
  \centering
  \includegraphics[width=0.9\textwidth]{./graf/enc-graph.jpg}
  \caption{Wykres stanów wyprowadzeń enkodera}
  \label{fig:encoder-graph}
  \end{figure}


\section{Przełącznik trybów}
  Jednym z założeń dodatkowych projektu jest możliwość przełączania SWIM między trybami lokalnym oraz sieciowym. 
  Aby zapewnić taką możliwość bezpośrednio z poziomu modelu wykorzystano przełącznik bistabilny. 

\section{Serwer}
\section{Punkt dostępowy}
\section{Strona internetowa}
\section{Interfejs komunikacyjny}
\section{Zasilanie}