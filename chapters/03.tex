\chapter{Część projektowa}
% \label{ch:wymagania-i-narzedzia} % nie wiem do czego to służy więc nie usuwam póki co

\section{System mikroprocesorowy}
  Ze względu na wymaganą komunikację sieciową projektu, zdecydowano się na użycie prostego mikrokontrolera z
  rodziny ESP32. Głównymi zadaniami użytego mikrokontrolera jest obsługa dwóch linii adresowalnych diod LED, obsługa 17 przycisków 
  wraz z enkoderem, kontrola podświetlenia LED oraz hosting strony internetowej.

  Ze względu na dostępność, popularność, bogatą dokumentację oraz niską cenę zdecydowano się na użycie mikrokontrolera 
  \textit{ESP32-S3-WROOM-1} firmy \textit{Espressif Systems}
  \\

  Własności układu
  \begin{itemize}
    \item dwurdzeniowy 32-bitowy mikroprocesor \textit{Xtensa LX7},
    \item 16MB pamięci flash,
    \item 16MB pamięci PSRAM,
    \item 512MB pamięci SRAM,
    \item 36 portów GPIO,
    \item komunikacja 2.4GHz WiFi (802.11b/g/n) oraz Bluetooth 5,
    \item wbudowana antena,
    \item wsparcie dla: I2C, LED PWM, USB Serial,
    \item zasilanie napięciem 3.3V.
  \end{itemize}

\clearpage
\section{Adresowalne diody LED}
  Projekt zakładał wizualizację wielu informacji na znaczących rozmiarów modelu Maszyny W.
  Przewidywana liczba pojedynczych diod LED przekracza 600. Oznaczać to może znaczące spadki
  napięcia na długiej linii LED.

  Podczas wstępnych testów linii 5V zaobserwowano zauważalny spadek jakości odwzorowania kolorów.
  \\

  Biorąc pod uwagę wyżej wymienione wymagania zdecydowano się na użycie adresowalnych diod 
  \textit{WS2815B-V1} firmy \textit{WORLDSEMI}.
  Wybrany model diod pozwala na indywidualne zadawanie wartości Red, Green oraz Blue każdej z nich.
  Wyższe napięcie zasilania eliminuje powstałe w wyniku spadków napięcia zmiany barwy.
  \\

  Własności elementu
  \begin{itemize}
    \item napięcie pracy 12V,
    \item obudowa 5050,
    \item montaż SMD,
    \item częstotliwość odświeżania 4kHz.
  \end{itemize}


\section{Przyciski}
  W celu zapewnienia pełnej obsługi Maszyny W potrzebne jest wykorzystanie 17 monostabilnych przycisków.
  Symulator w wersji podstawowej udostępnia użytkownikowi 16 możliwych sygnałów: 
  \textit{czyt, wys, wei, il, wyad, weak, pisz, przep, wel, wyl, dod, ode, wes, weja, wyak, wea}. Do wymienionych sygnałów 
  należy również wliczyć przycisk \textit{takt}, który zapewni możliwość wykonania zadanego rozkazu.
  \\

  Zdecydowano się na wykorzystanie przełączników \textit{MX Blue} firmy \textit{Cherry}. Według dokumentacji producenta
  wyżej wymieniony przełącznik klawiaturowy pozwala na 50 milionów kliknięć. Pozwoli to na długoletnią, niezawodną działalność.
  Wykorzystanie przycisków typu \textit{Blue} oferujących słyszalny odgłos kliknięcia przy wciskaniu. Daje to użytkownikowi
  informację zwrotną o stanie przełącznika.
  \\

  Wykorzystanie tak dużej liczby przycisków wymaga użycia multipleksera 16:4. Z uwagi na niską cenę, łatwą dostępność
  oraz niewielki wymiar wybrano model \textit{CD74HC4067M} firmy \textit{Texas Instruments}.


\clearpage
\section{Podświetlenie LED}
  Chcąc zapewnić przyjazny wizualnie wygląd modelu SWIM zdecydowano się na podświetlenie następujących elementów:
  \begin{itemize}
    \item licznik rozkazów,
    \item akumulator,
    \item rejestr instrukcji,
    \item rejestry A oraz S,
    \item logo Politechniki Śląskiej.
  \end{itemize}

  Wykorzystano do tego 12V linie diod LED.

  \textit{TODO skończyć tekst PODŚWIETLENIA LED}

\section{Enkoder}
  Do wprowadzania danych liczbowych w lokalnym trybie działania urządzenia zastosowano enkoder obrotowy z przyciskiem.
  Wykorzystanie enkodera:
  \begin{itemize}
    \item zwiększanie/zmniejszanie wartości liczbowych w zależności od kierunku obrotu,
    \item aktywacja trybu wprowadzania danych poprzez przytrzymanie przycisku,
    \item przełączanie między polami wprowadzania danych przy użyciu szybkich kliknięć.
  \end{itemize}
  Enkoder pozwala użytkownikowi w prosty i intuicyjny sposób sterować parametrami urządzenia bez korzystania z 
  interfejsu sieciowego.

\section{Przełącznik trybów}
  Jednym z założeń dodatkowych projektu jest możliwość przełączania SWIM między trybami lokalnym oraz sieciowym. 
  Aby zapewnić taką możliwość bezpośrednio z poziomu modelu wykorzystano przełącznik bistabilny. 