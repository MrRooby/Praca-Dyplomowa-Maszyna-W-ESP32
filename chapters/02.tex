\chapter{Przykładowa realizacja systemu wizualizacji}

\section{Założenia realizacyjne}
  Pierwszym etapem budowy systemu wizualizacji danych jest projektowanie, które powinno być poprzedzone przemyśleniami
  dotyczącymi możliwych realizacji oraz wyborem tych, które są optymalne i spełniają przyjęte założenia. Założenia dodatkowe
  nie są bezwzględnie wymagane, ale mogą poszerzyć tworzone rozwiązanie o alternatywne sposoby działania.
  \\


  Założenia podstawowe:
  \begin{itemize}
    \item dedykowany układ dla mikrokontrolera z odpowiednimi wyprowadzeniami,
    \item dedykowany układ dla LED-owych wyświetlaczy siedmiosegmentowych,
    \item konstrukcja kompletnego układu elektronicznego z układów mikrokontrolera, wyświetlaczy LED oraz przycisków,
    \item hosting strony internetowej bezpośrednio na urządzeniu wraz z przesyłem danych pomiędzy urządzeniem a stroną,
    \item wizualizacja informacji pobieranych ze strony internetowej na adresowalnych diodach LED,
    \item przesył informacji o wciśniętych przyciskach do strony internetowej,
    \item obsługa podświetlenia modelu Maszyny W diodami LED.
  \end{itemize}


  Założenia dodatkowe:
  \begin{itemize}
    \item lokalna symulacja Maszyny W na mikrokontrolerze,
    \item obsługa enkodera do wprowadzania wartości liczbowych.
  \end{itemize}

% \clearpage  % nowa strona 
\section{Schemat blokowy}
  Na rysunku poniżej (Rys. \ref{fig:schemat-blokowy}) przedstawiono przykładowy schemat blokowy systemu wizualizacji 
  interfejsu maszyny (SWIM).
  Schemat uwzględnia zarówno bloki spełniające założenia podstawowe, jak i dodatkowe. 
  \\


  \begin{figure}[H]
  \centering
  \includegraphics[width=0.9\textwidth]{./graf/schemat-blokowy-ogolny.pdf}
  \caption{Schemat blokowy SWIM}
  \label{fig:schemat-blokowy}
  \end{figure}